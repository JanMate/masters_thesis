\section{Android specific UI widgets - Material Design}\label{sec:android-specific-ui-widgets}
\textbf{Material Design} is a concept that was introduced by Google company in 2014.\cite{materialDesignArticle}
All Flutter User interface is built on this concept.
Material Design contains components that interactively builds blocks for creating a user interface.
The components are following:
\begin{itemize}
    \item App bar,
    \item Bottom navigation,
    \item Buttons,
    \item Floating Action button,
    \item Cards,
    \item Chips,
    \item Dialogs,
    \item Lists,
    \item Pickers,
    \item Progress indicators,
    \item Sliders and
    \item Tabs.\cite{materialDesign}
\end{itemize}

\textbf{Scaffold} represents a new rendered screen which allows placing various elements.
\textbf{Scaffold} is possibly fully customize and change by the requirements.
If we emphasize the simplicity, we can use the current interface and define:
\begin{itemize}
    \item AppBar,
    \item floatingActionButton, and
    \item bottomNavigationBar.
\end{itemize}
\textbf{AppBar} represents the header of the screen and usually contains a title and action buttons.
\textbf{FloatingActionButton} is a concept of buttons that a user uses the most time, and thus they are in the thumb zone.
\textbf{BottomNavigationBar} is a concept of a horizontal menu that is in the thumb zone in the footer of the screen and it allows creating the main well-arranged menu.
