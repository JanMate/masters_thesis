\section{Material Design}\label{sec:android-specific-ui-widgets}
Material Design is a concept that was introduced by Google company in 2014.~\cite{materialDesignArticle}
All Flutter user interface is built on this concept.
Material Design contains components that interactively build blocks for creating a user interface.~\cite{materialDesign}
The components are the following:
\begin{itemize}
    \item App bar,
    \item Bottom navigation,
    \item Buttons,
    \item Floating Action button,
    \item Cards,
    \item Chips,
    \item Dialogs,
    \item Lists,
    \item Pickers,
    \item Progress indicators,
    \item Sliders,
    \item Tabs.
\end{itemize}

\subsection{Scaffold}\label{subsec:scaffold}
Scaffold represents a new rendered screen that allows placing various elements.
The Scaffold is possibly fully customized and change by the requirements.
If developers emphasize simplicity, it is able to use the current interface and define:
\begin{itemize}
    \item AppBar,
    \item FloatingActionButton,
    \item BottomNavigationBar.
\end{itemize}
AppBar represents the header of the screen and usually contains a title and action buttons.
FloatingActionButton is a concept of buttons that users use the most time, and thus these buttons are in the thumb zone.
BottomNavigationBar is a concept of a horizontal menu in the thumb zone in the footer of the screen, and it allows the creation of the main well-arranged menu.~\cite{flutterBook}
\newline

