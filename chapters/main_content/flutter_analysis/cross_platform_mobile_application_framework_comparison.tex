\section{Cross-platform Mobile Application Framework Comparison}\label{sec:cross-platform-mobile-application-framework-comparison}
This section describes a comparison of cross-platform mobile application frameworks.
These frameworks are \textbf{Flutter}, \textbf{React Native}, \textbf{Ionic} and \textbf{Xamarin}.

\textbf{Flutter} is a framework and was established by Google Inc., in 2017, and since that time, its popularity is still increasing.
It means, \textbf{React Native} is widespread and most popular.
\textbf{React Native} looks like the leading competitor and is based on JavaScript.~\cite{flutterVsReactNativeNevercodeIo}
\textbf{Ionic} has not such excellent performance as \textbf{Flutter} and \textbf{React Native}.~\cite{crossPlatformFrameworokComparation}
\textbf{Xamarin} is based on C\# with .NET extension from Microsoft, and thus it was ommited.
\textbf{Flutter} is based on Dart, which was introduced by Google Inc., in 2011.
It is a type-safe and object-oriented programming language similar to Java.~\cite{dartTypeSystem}
Between the advantages of \textbf{Flutter} belong:
\begin{itemize}
    \item Hot Reload, i.e., allows fast coding,
    \item One codebase: Development for two mobile platforms,
    \item Up to 50~\% less testing,
    \item Faster app development,
    \item User-friendly designs,
    \item Perfect for MVPs,
    \item Less Code.
\end{itemize}
These advantages are available from~\cite{flutterVsReactNativeHackrIo}.
\newline
\newline
There is a comparison summary:
\begin{itemize}
    \item \textbf{Flutter} has the best performance and uses proprietary Widgets to create a user interface instead of HTML, CSS and JavaScript.
    \item \textbf{React Native} is the most popular because it is based on JavaScript.
    \item 98~\% of code in \textbf{Ionic} is reusable.
    \item \textbf{Xamarin} allows reusing code in Xamarin.Forms, but requires a paid development environment for business purposes.
\end{itemize}