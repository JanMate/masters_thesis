\section{Cupertino Library}\label{sec:ios-specific-ui-widgets}
The \textbf{Flutter} developers have decided to incorporate familiar elements from the iOS platform into the library for using these elements in the \textbf{Flutter} user interface.
The reason is that it would be easy to use for development, focusing on iOS devices.
This library calls \textbf{Cupertino} and the name was established by the Apple company headquarters building in Silicon Valley, California, in the United States of America.
Familiar elements from iOS devices are part of the library because the iOS users are using them.~\cite{cupertino}
This library contains the following elements:
\begin{itemize}
    \item Cupertino Action Sheet,
    \item Cupertino Activity Indicator,
    \item Cupertino Alert Dialog,
    \item Cupertino Button,
    \item Cupertino Context Menu,
    \item Cupertino Date Picker,
    \item Cupertino Dialog,
    \item Cupertino Navigation Bar,
    \item Cupertino Page Scaffold,
    \item Cupertino Picker,
    \item Cupertino Slider,
    \item Cupertino Switch,
    \item Cupertino Tab Scaffold.
\end{itemize}
