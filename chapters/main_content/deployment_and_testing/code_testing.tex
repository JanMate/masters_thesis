\section{Code testing}\label{sec:code-testing}
Testing má v SW develpment zásadní využití.
Pokud Development probíhá v rámci agilní metodiky, znamená to, že se jedná o menší dodávky sw, které obvykle zahrnují konkrétní funkcionalitu nebo bug fixing.
V případně takovýchto zásahů do již nasazené aplikace je nutné alespoň částečně otestovat jejich správnou funkcionalitu, abychom předešli budoucím problémům.
Největší problém nastává tam, kde vznikají závislosti.
Tedy pokud funkcionalita už na začátku pracuje špatně, bude i dalším fukcionalitám postavených na ní dávat špatné výsledky a může to způsobit dokonce nestabilitu application.
Výhodou je že tyto testy lze spouštět opakovaně a při změnách v kódu jednoduše upravovat.
Testing je tedy fáze součástí SW develpmentu, kteŕa lze provozovat automatizovaně v rámci CI a CD.

Tato kapitola dále popisuje typy testů a funkcionality v aplikace, které jsou těmito testy pokryté včetně popisu, jak se mají dané funkcionality chovat v konkŕetnich situacích.

\subsection{Unit testing}\label{subsec:unit-testing}
Tato sekce se věnuje  UnIT testům.
Unit testy ve Flutteru lze spouštět příkazem z console prompt a díky tomu je lze snadno včlenit do fáze v CI a CD, kde se spustí automaticky.
Pro UnIT testing se ve Flutteru používání následující knihovny:
UnIT test
Mockito
Bloc\_test
And so on.

Unit testy v této aplikaci pokrývají následující funkcionality:
\begin{itemize}
    \item
\end{itemize}
%TODO:

\subsection{Integration testing}\label{subsec:integration-testing}
%TODO:

Integrační testy v této aplikaci pokrývají následující funkcionality:
\begin{itemize}
    \item
\end{itemize}

\subsection{Mocker}\label{subsec:mocker}
Mocker for flight drone simulations.
Mocker is written in Python a slouží k zasílání messages to backend endpoint for testing purpose.
Mocker přijímá parametry se serial number and allows to specify a take off position.
It allows to mock more drones concurrent, too.
Výhodou tohoto Mockeru je, že při vývoji nepotřebujeme mít dron se zařízením přímo ve vzduchu, ale pro testovací účely nám pro nasimulování chování stačí právě tento Mocker.
