\section{Code testing}\label{sec:code-testing}
Testing has in the Software development the key usage.
If the development runs according to an agile methodology, it means that it is about small regular deliveries of software pieces that includes the certain functionality or bug fixing.
In case of these interferences of the already deployed application is needed at least to partly test its correct functionality to prevent potential problems in the future.
The biggest problem occurs where dependencies creates.
So, if the functionality falls down at the beginning, the functionalities depending on the root functionality will give a bad results, and it can cause even instability of the application.
The advantages is the tests are able to run repeatedly and in case of changes we can easily change them.
So, the testing is a phase in the Software development that is allow to automation in the Continuous Integration and Continuous Delivery.

In addition, this chapter describes kinds of tests and functionalities in the application, which are covered by these tests.
It includes a description how the given functions should behave in the certain situations.


\subsection{Unit testing}\label{subsec:unit-testing}
This section emphasizes to Unit tests.
In Flutter, Unit tests are able to run by \textit{flutter test} command from the console prompt.
Thanks this, it is easy to incorporate them into Continuous Integration and Continuous Delivery phase where the test care run automatically.
There are libraries used for Unit testing in Flutter:
\begin{itemize}
    \item test,
    \item flutter\_test,
    \item Mockito,
    \item Bloc\_test and
    \item others.
\end{itemize}

Unit tests covers the following functionalities in the application:
\begin{itemize}
    \item %TODO:
\end{itemize}


\subsection{Integration testing}\label{subsec:integration-testing}
This section emphasizes to Integration test.
These tests verify if the alone parts tested by unit tests can cooperate together and work well as a whole.
We have implemented on integration test because we had no time to deal with this kind of tests.


\subsection{Mocker}\label{subsec:mocker}
Mocker is used for flight drone simulations.
Mocker is written in Python and sends messages to backend endpoint for testing purpose.
Mocker expects parameters serial number and allows to specify a take off position (optional parameter).
It allows to mock more drones concurrently, too.
The advantages of this Mocker is I do not need to have a flying physical drone with attached Dronetag device in the air during the development.
For a simulated behavior is enough to run this Mocker.
