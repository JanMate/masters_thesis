\section{Configuration files}\label{sec:configuration-files}

For using \textbf{Continuous Integration} and \textbf{Continuous Delivery} is needed to set right configuration.
Therefore, there is a need to create alone configuration files for every environment.
For this purpose we used \textbf{environment\_config} Flutter library, that allows easy management and provides the generator.
This generator allows to use these settingsin the codebase by the given parameters.
The setting of these variables is kept by the configuration YAML file, where you choose required configuration and the generator makes available it for you via an interface in the codebase.
Here I have to notice, that the cross-platform frameworks like Flutter allow to set separate configuration for every platforms alone - iOS and Android.
Unfortunetely, there are even cases when the setting applies as similar as on the both of the platforms.
For example, it can be URL (Uniform Resource Locator) addresses, client ID, client secrets and a path to log file.
If you need to distinguish a configuration for each of the platform alone, you will perform this change in the specific bundle.
A typical example could be differentiate setting of Google Maps widget, where is needed to set activated Google Maps API key.
Although, these platforms have different architecture, so its configuration is different.

Immediately at the beginning of the development phase, when a new member joins the team, he must to clone \textbf{mobile-app} folder from the git repository on BitBucket server.
After that, it must be added files \textbf{auth\_config.json} and \textbf{google\_map\_api\_key.json} in directory \textbf{assets/config}.

\textbf{auth\_config.json} file in the format:
% @formatter:off
\begin{verbatim}
{
    "clientId": "...",
    "clientSecret": "..."
}
\end{verbatim}
% @formatter:on


This file is used for the security of the client access and in case it would be stolen from one environment, it will not be applicable in another environment.

\textbf{google\_map\_api\_key.json} file is in the format:
% @formatter:off
\begin{verbatim}
{
    "apiKey": "..."
}
\end{verbatim}
% @formatter:on

This file is for the access to the information, which are provided by \textbf{Google Cloud Platform API}.
In our project, we use Google Maps API for downloaing map styles, data about a searching place, and the Autocomplete function that suggests close result by the given substring.

All mentioned files meet the JSON format standard specification.


\subsection{Develop environment}\label{subsec:develop-environment}
The configuration files in this project contains setting for local running platform, that is launched as Kubernetes cluster.
The part of the cluster is a docker container with the backend implementation.
More information about platform you can find in the chapter Dronetag web infrastructure \ref{ch:dronetag-web-infrastructure}.


\subsection{Test environment (Staging)}\label{subsec:test-environment}
Test environment is called Staging and is used for testing and debugging new functionalities and feature, and \textbf{Software Quality Assurance}.\cite{sqa}
In this environment, we can simulate arbitrary model situation and check how the application behaves in these edge case situations, and if it is stable in each of the scenario.
In addition, it is posible to verify the fact if a fault occours, that the clients can adequately respond to this situation and catch created exceptions.


\subsection{Production environment}\label{subsec:production-environment}
Production environment is used for real application operations and it should contain no errors created during development phase.
This environment is crucial for the Dronetag company business and the customers using the product have the feeling about flawless application any time when they will use it.
Configuration files in this environment contain the connection settings to Production server, where is operating the cluster with the backend instance.
