\chapter{Related projects}\label{ch:related-projects}
This chapter aims to related projects...

At first, I would like to talk about the solutions in the world especially using of drones in the industry.
For example, Amazon Inc. company is working on a proof of concept which can use drones to delivery a package ordered by a customer.
This news was published in Forbes magazine by Jillian D'Onfro.\cite{amazonArticle}
The customer could be more satisfied with this shopping because the time of delivering can be cut on as the tiny part of the time as now.
Even Amazon has decided to produce their own drone products.
One of them is Prime Air delivery drone.

To make the delivery process easier, Amazon is developing autonomous drones application to minimize costs.
Amazon using/usage/application

autonomous drones application

Potentially, it could be possible to use drones to delivery the medical materials in case of big accident.
Especially, after a hurricane, fight conflict, or even ...


This chapter aims to related projects that deal with safe drone operation especially web and mobile applications to subscribe the current information about flight restrictions.
It involves showing restricted area to flying, guide with advisories and terms which a pilot has to meet during his flight.
...

\section{AirMap}\label{sec:airmap}
AirMap is the~main concurent application comparable with Dronetag platform.%\cite{}

\section{AisView}\label{sec:aisview}
AisView is abbreviation of Airspace Information System View and is for observation of flight zones and restricted area.
TODO

\section{Fly carefully (L{\' e}tejte zodpov{\v e}dn{\v e})}\label{sec:fly-carefully}
TODO

\section{Project in Dronetag s.r.o. company}\label{sec:project-in-dronetag-s.r.o.-company}
describe the complex concept how can common user can use it for (Dronetag Pro and Mini)


TODO integration to foreign API endpoints

