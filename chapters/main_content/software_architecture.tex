\chapter{Software architecture}\label{ch:software-architecture}
This chapter describes a software architecture of the mobile application.
At the beginning, let me clarify the difference between the software design and architecture.
In Software development, Software architecture means the design of application from the higher perspective than Software design.
It means that when you start with new application project, it is important to arrange the project into small alone parts.
These parts should meet the encapsulation and comprehensive rule.
The libraries offering generics classes that include already implemented logic help to simplify and do code more readable thanks to the standards.
For better understanding, I will describe it on the following example.
The architecture usually means a division into bigger self-driven parts.
Typically, there are the concepts like MVC (Model-View-Controller) or their extended descendents like MVP (Model-View-Presenter) or MVVM (Model-View-ViewModel).
The newer famous concept in Flutter is Bloc.
I have already explained what the Bloc is in the chapter~\ref{ch:analysis-of-flutter} Flutter Analysis so I will not clarify it anymore.
On the other hand, I would like to devote to talk about the difference between common Bloc and its extended class Hydrated Bloc.
Both of them using the sink and the stream concept working on the stream subscription principle.
The Hydrated bloc offers a persistent storage storing and loading with fromJson and toJson methods.
It helps to have an application working more fluently than alone loading data via the network every time when a user needs them.

In opposite, Software design means the division of the code into parts that describe implementation details closely.

\section{Dependencies}\label{sec:dependencies}
Flutter has a huge database of dependencies that are maintained by various developers.
The database is called \textbf{pub.dev}~\cite{pubDev} and offers to their users a summary for every dependency with the score analysis.
This analysis consists of Popularity, Health, Maintenance and final Overall.
The dependencies used in this mobile application are separated into following categories:
\begin{itemize}
    \item Icons,
    \item State management,
    \item NoSQL storage,
    \item Tools,
    \item Firebase,
    \item API,
    \item UI,
    \item Map and
    \item Generator.
\end{itemize}

\subsection{Icons}\label{subsec:icons}
Icons contains \textbf{cupertino\_icons} that provides iOS Cupertino Icons by Apple Inc.
\textbf{Material Design icons} are already part of the \textbf{Flutter SDK} library.

\subsection{State management}\label{subsec:state-management}
State management consists of \textbf{flutter\_bloc} and \textbf{hydrated\_bloc}.

\subsection{NoSQL storage}\label{subsec:nosql-storage}
NoSQL storage contains \textbf{hive} and \textbf{hive\_flutter}.
These dependecies ensure NoSQL storage that are for persitent data in offline mode and store things like JWT token.\cite{jwtToken} %TODO: add cite

\subsection{Tools}\label{subsec:tools}
Tools consist of:
\begin{itemize}
    \item \textbf{intl} - it ensures internationalization,
    \item \textbf{equatable} - it ensures that object extending it can be easily comparable by properties of that object,
    \item \textbf{flutter\_svg} - it ensures drawing SVG image as a widget,
    \item \textbf{flutter\_i18n} - it ensures Internationalization by the user phone settings,
    \item \textbf{path\_provider} - it ensures finding commonly used locations on the filesystem,
    \item \textbf{print\_lumberdash} - it ensures a convenient debug printing of an error message,
    \item \textbf{get\_it} - it allows using of Dependency Injection in code - it supports IoC principle,\cite{iocPrinciple}
    \item \textbf{dartz} - it allows using functional programming in Dart,
    \item \textbf{url\_launcher} - it ensures launching a URL in the mobile platform,
\end{itemize}

\subsection{Firebase}\label{subsec:firebase}
Firebase contains \textbf{firebase\_core} and \textbf{firebase\_messaging}.
These dependecies ensure connecting to Firebase Cloud and allowing push notifications.

\subsection{API}\label{subsec:api}
API consists of:
\begin{itemize}
    \item \textbf{http} - it ensures making HTTP requests,
    \item \textbf{json\_annotation} - it ensures definition of annotations used by JSON serialization and deserialization,
    \item \textbf{data\_connection\_checker} - it checks for an internet connection by opening a socket,
    \item \textbf{retrofit} - it ensures type conversion \textbf{dio} client generator to create HTTP requests,
    \item \textbf{pretty\_dio\_logger} - it ensures a convenient debug printing of HTTP requests and responses,
    \item \textbf{dio} - it offers support of Interceptors, Global configuration, FormData, Request Cancellation, File downloading, Timeout etc.,
    \item \textbf{flutter\_screenutil} - it ensures adapting screen and font size.
\end{itemize}

\subsection{UI}\label{subsec:ui}
UI consists of:
\begin{itemize}
    \item \textbf{sliding\_up\_panel} - it offers using done Sliding up panel,
    \item \textbf{animations} - it contains pre-canned animations for commonly-desired effects,
    \item \textbf{flutter\_sfsymbols} - it supports all iOS and Android devices.
\end{itemize}

\subsection{Map}\label{subsec:map}
Map consists of:
\begin{itemize}
    \item \textbf{google\_maps\_flutter} - it provides a Google Maps widget,
    \item \textbf{location} - it handles getting location on Android and iOS,
    \item \textbf{google\_maps\_webservice} - it offers using Google Web Service API that including Geocoding, Places, Directions, and so on.
\end{itemize}

\subsection{Generator}\label{subsec:generator}
Generator consists of:
\begin{itemize}
    \item \textbf{hive\_generator} - it generates storable object by Hive annotations,
    \item \textbf{retrofit\_generator} - it generates implementation of HTTP requests due to set annotations,
    \item \textbf{bloc\_test} - it generates testing blocs for unit testing,
    \item \textbf{mockito} - it generates Mocks for unit testing,
    \item \textbf{pedantic} - it ensures a linter that looking after to meet best practices,
    \item \textbf{build\_runner} - it provides a concrete way of generating files using Dart code,
    \item \textbf{json\_serializable} - it generates serializable object by JSON annotation.
\end{itemize}

\section{Bloc and Hydrated Bloc}\label{sec:bloc-and-hydrated-bloc}
In the beginning, we counted on all of BLoCs will be the classic one.
After short time, we have found out the storable concept is more convenient for this purpose.
So, some following we changed to descendants of HydratedBloc.

\subsection{User Profile Bloc}\label{subsec:user-profile-bloc}
At the beginning, User Profile Bloc had extended from common Bloc but after increasing data traffic we decided to change it to Hydrated Bloc.
The reason is simple.
User properties do not change so often.
Due to it better to store the data in phone and update them when it changes.

\subsection{Aircraft Operation Bloc}\label{subsec:aircraft-operation-bloc}
%TODO
\subsection{Device Operation Bloc}\label{subsec:device-operation-bloc}
%TODO
\subsection{Flight OperationBloc}\label{subsec:flight-operationbloc}
%TODO

\section{JsonSerializable classes}\label{sec:jsonserializable-classes}
Json Serializable interface offers generating of  methods for serialization and deserialization classes that are annotated by Json Annotation.
These classes are converted and created the methods via the \textbf{build runner} library.
The advantage is a developer must not code these methods by hand, and he can re-generated every time it is needed to change the class structure.

\subsection{Request classes}\label{subsec:request-classes}
%TODO
\subsection{Response classes}\label{subsec:response-classes}
%TODO

\section{Dio class}\label{sec:dio-class}
Due to need of using Interception and fill prepared HTTP requests with Authentication Bearer we defined our a custom \textbf{Dronetag Dio} class.
It initializes \textbf{Pretty Dio Logger} to make clear debug printing and Token Interceptor that extends \textbf{Interceptors Wrapper}.
