\section{AisView}\label{sec:aisview}
\textbf{AisView} (or its simplified DroneView) is an application managed bye the \textbf{Air Navigation Service of the Czech Republic} before starting flight purposes.
There are restrictions, map zones, \textbf{NOTAMs} (Notice To Airmen), and planned drone flights on the map.
It has no real-time drone location data.

Users set flight a time range (from and to), and after they will be able to see relevant information related to them in the map during the time range.
Besides, the users must manually click to shown zones to see more information.

The users can see a map covered with additional information about the weather (especially a layer of data from CHMI radar~\cite{chmi}) or choose a type of map data (ICAO~\cite{icao}, orthographic, tourist).

The user interface is quite unfriendly and non-intuitive in some cases.
Abbreviations sometimes label the buttons, and, certainly, the application is rather for professionals.
A layperson can have problems with this application and does not have to understand its possibilities.
Neither the \textbf{DronView} does not support such usable functionalities for users than the \textbf{AisView}.

After logging in, users can add registration numbers of their devices and announce drone flight plans.
Also, there are options to store various minor showing map preferences and other details for the logged users.

\textbf{AisView} primarily provides especially a web variant, that can use on the computer and mobile phones.
There is a variant for mobile phones, too.
However, it is quite limited and looks obsolete and does not include all system functions.
\newline
\newline
Advantages:
\begin{itemize}
    \item It provides data from the verified source.
    \item The application is used by pilots in the daily routine and is stable.
    \item A user can announce his flight plan to authorities from the application.
    \item It is the main source of the \textbf{Air Navigation Service of the Czech Republic}, so it is always up-to-date.
\end{itemize}
Disadvantages:
\begin{itemize}
    \item The application user interface looks obsolete, and the user experience is often unsatisfactory for the public.
    \item It can see planned drone flights, except real-time drone positions in the application.
    \item The geographic range is only for the Czech Republic.
    The data of other countries are not visible.
\end{itemize}
