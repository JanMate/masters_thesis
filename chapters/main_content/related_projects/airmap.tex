\section{AirMap}\label{sec:airmap}
AirMap is the~main competition application comparable with Dronetag platform.\cite{airMap}
This application shows flight zones and restricted areas around the whole Europe.
In addition, it allows easy flight planning.
Unfortunately, this planning has no central management, thus this solution is insufficient for ensuring the aviation safety.
AirMap ecosystem is able to divide into some branches:
\begin{itemize}
    \item AirMap UTM,
    \item AirMap Pilot,
    \item AirMap Enterprise and
    \item AirMap Developers.
\end{itemize}

\subsection{AirMap UTM}\label{subsec:airmap-utm}
AirMap UTM is a web application used for aviation flight dispatchers.
It allows approving users' planned flights in restricted areas (for example CTR) by authorities.
Manager of this zone can show a detail of application form and request him to fill information about flight.

Advantages:
\begin{itemize}
    \item An operator can see active flight plans and others flight operation and its altitudes (airplanes and helicopters) in the same time.
    \item An operator can contact the pilot via SMS.
    \item An operator can see a DJI drone telemetry of a pilot in case that the pilot is using AirMap mobile application in-flight mode.
    \item Whenever an operator can create a restricted zone for flights, he can notify a pilot via SMS.
    \item An operator can filter received application forms.
    \item It is used to using in the USA, the Czech Republic, Switzerland, Japan, New Zealand, \textellipsis
\end{itemize}
Disadvantages:
\begin{itemize}
    \item It offers no on-premise solution.
    \item It is quite trivial implementation so far.
\end{itemize}


\subsection{AirMap Pilot}\label{subsec:airmap-pilot}
The main goal of this application is to force pilots to fly carefully in according to valid rules of their location.
Concurrently, it provides an official authorization to the restricted airspaces
In addition, the application allows controlling DJI drones instead of the official DJI GO mobile application.

After the launch of the application, the flight map is loaded with map data from Mapbox~\cite{mapBox} and a current user location is focused in the map with available zones around.
There are immediately seen labels of every zone and advisories for flights in the given country in the map.
The suggestion list does not look intuitive.
In the bottom panel with advisories is also visible a temperature and wind speed in the location.
Unfortunately, it misses various values for arbitrary flight levels.
That is it all the functionality for a user, nothing else.

Advantages:
\begin{itemize}
    \item It is used to using in the USA, the Czech Republic, Switzerland, Japan, New Zealand, \textellipsis
    \item The application shows flight zones of all countries (data from EUROCONTROL EAD).\cite{eurocontrol}
    \item Pilots are able to get authorization of airspace (LAANC) in real time.
    \item Flights can be also planned in the future, it is not allowed to change them after that. (remove only)
    \item In case of planning mission in the web application, I can see it in the mobile application in a minute.
    But the mission date does not match.
    The same case, when I cancel it in the mobile application, I can see it will disappear in the mobile and web application in a minute, too.
    \item It contains Geofencing alerts - it notifies a pilot about the entrance int the managed flight zone.
\end{itemize}
Disadvantages:
\begin{itemize}
    \item V mobile application is missing the option to click on the requested zone and do anything with it or get information.
    It is quite inconvenient in case of many intersected zones in one point.
    The web application allows it.
    \item Many times I have got a timeouted request and sometimes something was showing inconsistent.
    For example, I have not seen any rules in Flight Briefing section during planning first flight mission.
    In second attempt, I have already seen it.
    Generally, all loading or submitting takes a long time.
    \item It can provide no flight log export nor other data.
    \item It supports no flight plan storage for the future nor repeating.
    \item It contains a few amateur bugs.
    For example, in the web application, if I draw a trajectory and press ESC key on the keyboard, I will not able to draw again without need to choose a tool and then the default again.
\end{itemize}


\subsection{AirMap Enterprise}\label{subsec:airmap-enterprise}
AirMap Enterprise is a dashboard as similar as AirMap UTM, but it is specialized to the narrow group of users.
It is not able to find accurately, what is exactly means, but the screen looks as similar as AirMap UTM. For example, application form approval, real-time telemetry, pilot contacting.
It is a kind of fleet management solution.


\subsection{AirMap Developers}\label{subsec:airmap-developers}
AirMap Developers is a portal offering a documentation about AirMap API and SDK, that is able to use for integration the AirMap services into individual solutions.\cite{airMapDevelopers}
