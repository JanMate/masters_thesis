\section{Altitude Angel}\label{sec:altitude-angel}
Altitude Angel ecosystem can divide into some branches:
\begin{itemize}
    \item Guardian UTM,
    \item Guardian Mobile and
    \item Drone Safety Map.
\end{itemize}

\subsection{GuardianUTM}\label{subsec:guardianutm}
There is only one way to get information on the official documentation.\cite{altitudeAngel}


\subsection{Drone Safety Map}\label{subsec:drone-safety-map}
It is a mobile and web application for drone pilots connected to GuardianUTM. Its goal is to show zones and flight missions of all pilots.

Advantages:
\begin{itemize}
    \item The application shows flight zones in all countries.
    \item Aside from zones, it also shows hazardous objects like high-voltage wires, schools, police officers, and gas stations.
    Maybe, it is crowded with information and map data was not clear.
    \item It supports CAA UK that integrates its solution into the application of NATS Drone Assist.
\end{itemize}
Disadvantages:
\begin{itemize}
    \item The registration is pain with an additional password setting via e-mail.
    \item The flight plan creating contains annoyed text fields (flight title, description, timezone).
    The flight plan history does not exist.
    \item There is no chance to share telemetries in real-time (for example, connection with drones DJI).
    \item It does not allow the planning of complicated zones than circles.
    \item It is difficult to determine a reserved zone for flight against restricted zone by the authorities in User Interface.
    \item Completely, the mobile and web application is quite weak with UI beside AirMap.
    For example, it shows terms \& conditions before every flight.
    \item The mobile application has crashed during mission planning and searching, too.
\end{itemize}
