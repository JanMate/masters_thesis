\subsection{IO}\label{subsec:io}
IO directory obsahuje componenty, které zajišťují načítání dat z vzdálených zdrojů.
Zde byl kladen důraz na encapsulaci a comprehensition, proto jsme pro to vytvořili samostatné directory.
Tento directory je rozdělen na:
\begin{itemize}
    \item Model,
    \item Repositories and
    \item Services.
\end{itemize}
Tyto vnořené directories jsou detailněji popsány v dálších odstavcích.

\textbf{Model}~directory~contains objects that~are produced by~API endpoint of~Dronetag backend.
It is divided to \textbf{request} and \textbf{response} classes.
Model obsahuje takzvané entity, které slouží jako datové objecty.
Jejich hlavní účel je tedy držet data konsistentním způsobem.
Pro usnadnění práce pro mapování na data, která aplikace obdrží ze vzdálených datových zdrojů se stará Serializace.
Serializaci zajišťuje rozhraní JsonSerialization, která generuje serializované objecty podle zadané konfigurace.
Tato konfigurace se provádí prostřednictvím Json Annotation interface ve stejnojmenné knihovně.

\textbf{Repositories} consists of a repository for each of functional part.
Každá část obsahuje metody pro práci s jednotlivými entitami a zprostředkovává předávání dat ze vzdálených zdrojů.
Součástí každého repository je služba, která zajišťuje příslušné operace, které se mají provést s danou entitou z Model directory.
Služby jsou blíže popsané v dalším odstavci této sekce.
Repositories obsahuje následující třídy:
\begin{itemize}
    \item User Repository,
    \item Aircraft Repository,
    \item Device Repository,
    \item Flight Repository,
    \item
\end{itemize}
%todo

\textbf{Services} directory obsahuje podadresář \textbf{api}.
Tento podadresář obsahuje třídy, které zajišťují přijímání a odesílání dat na API endpointy, které jsou součástí Dronetag infrastruktury.
API rozhraní v této infrastruktuře zajišťuje Backend instance a Live Service.
Proto je také \textbf{api} directory rozdělen na:
\begin{itemize}
    \item Dronetag Web Rest Client,
    \item Dronetag Live Service Client,
    \item Auth Config,
    \item Google API Key Reader and
    \item Dronetag Dio.
\end{itemize}
Dronetag Web Rest Client slouží k načítání a posílání dat do webové platformy, konkrétně tedy backendu.
Obsahuje metody:
\begin{itemize}
    \item TODO
\end{itemize}

Dronetag Live Service Client slouží k načítání live dat z Live Service.
Důvodem rozdělení rozdělení na two clients je zpřehlednění a důvod, proč je i webová infrastruktura rozdělena na tyto 2 služby je popsán v kapitole Dronetag web infrastructure \ref{ch:dronetag-web-infrastructure}.
Teto client obsahuje metody:
\begin{itemize}
    \item TODO
\end{itemize}

Auth Config, Google API Key Reader and Dronetag Dio zajišťují jiný mechanismus a neslouží k přímému poskytování dat ze vzdálených zdrojů.
Avšak se nám hodilo ponechat tyto třídy pohromadě spolu s Rest clienty, jelikož spolu úzce souvisí.
Auth config obsahuje metody ...
Google API Key Reader obsahuje metody ...