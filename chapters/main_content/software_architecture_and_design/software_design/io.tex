\subsection{IO}\label{subsec:io}
IO directory contains components, that ensures loading data from remote sources.
Here we emphasizes to the encapsulation and comprehension, thus we have created alone directory..
This directory is divided to:
\begin{itemize}
    \item Model,
    \item Repositories and
    \item Services.
\end{itemize}
These nested directories are described in more detail in the following paragraphs.

\textbf{Model}~directory~contains objects that~are produced by~API endpoint of~Dronetag backend.
It is divided to \textbf{request} and \textbf{response} classes.
Model contains entities, that are used for data objects.
So, their main purpose is keep data in the consistent way.
Serialization ensures the facility mapping data to objects, that the application gets from the remote data sources.
Serialization ensures the interface \textbf{JsonSerialization}, that generates serialized objects by the given configuration.
This configuration is processed via \textbf{Json Annotation} interface in the library with the same name.

\textbf{Repositories} consists of a repository for each of functional part.
Every part contains methods for laboring with individual entities and mediates passing data from remote sources.
The part of every repository is a service that ensures relevant operations, that will process with the given entity from Model directory.
The services are closely described in the next paragraph in this seciton.
The Repositories directory contains the following classes:
\begin{itemize}
    \item User Repository,
    \item Aircraft Repository,
    \item Device Repository,
    \item Flight Repository,
    \item Search Repository and
    \item Zone Repository.
\end{itemize}

\textbf{Services} directory contains a subdirectory \textbf{api}.
This subdirectory contains classes that ensures receiving from and sending data to API endpoints, that are the part of the Dronetag infrastructure.
The API interface in this infrastructure is ensured by the Backend instance and Live Service.
Thus, \textbf{api} directory is divided into:
\begin{itemize}
    \item Dronetag Web Rest Client,
    \item Dronetag Live Service Client,
    \item Auth Config,
    \item Google API Key Reader and
    \item Dronetag Dio.
\end{itemize}
\textbf{Dronetag Web Rest Client} is used for laoding and sending data o the web platform, specifically the backend.
This client contains methods:
\begin{itemize}
    \item fetchMyUserProfile() - it returns a user profile data,
    \item authorize(String email, String password) - it authorizes a user via a JWT token endpoint,
    \item registerUser(RegisterRequest request) - it registers a new user,
    \item fetchFlightInfo(String id) - it returns a flight data by the given id,
    \item fetchAircrafts() - it returns all aircrafts that a user has granted access to see them,
    \item fetchAircraft(String uuid) - it returns an aircraft detail by the given id,
    \item addAircraft(String name, String uasOperatorID, String modelID, double weight) - it adds a new aircraft by the given parameters,
    \item updateAircraft(String uuid, String name, String modelID, double weight) - it updates an aircraft detail by the given uuid,
    \item deleteAircraft(String uuid) - it deletes an aircraft by the given uuid,
    \item fetchDevices() - it returns all devices that a user has granted access to see them,
    \item fetchDevice(String uuid) - it returns a device detail by the given uuid,
    \item addDevice(String name, String serialNumber) - it adds a new device with the given parameters,
    \item deleteDevice(String uuid) - it deletes a device by the given uuid,
    \item updateUserProfile(String uuid, UserProfile user) - it updates a user detail by the given uuid with the user parameter,
    \item fetchMyFlightStat() - it returns user's flight statistics,
    \item deleteFlight(String uuid) - it deletes a flight by the given uuid,
    \item fetchZones(String viewport) - it returns all zones by the given viewport,
    \item fetchZone(String uuid) - it returns a zone detail by the given uuid and
    \item passwordReset(String email) - it resets a user's password by the given e-mail address.
\end{itemize}

\textbf{Dronetag Live Service Client} is used for loading live data from Live Service.
The reason to divide Rest client into two clients is well-arrangement, and the reason why is the web infrastructure is separated into these two services, is described in the Dronetag web infrastructure chapter~\ref{ch:dronetag-web-infrastructure}.
This client contains methods:
\begin{itemize}
    \item fetchAllTelemetries() - it returns all last live telemetries about all Drones in airspace,
    \item fetchDeviceDetail(String serialNumber) - it returns a device detail by a given serial number and
    \item fetchTelemetryDetail(String serialNumber) - it returns all live telemetries about a drone by a given serial number.
\end{itemize}

\textbf{Auth Config}, \textbf{Google API Key Reader} and \textbf{Dronetag Dio} ensures another mechanism and is used for the direct providing data from the remote sources.
However, it was suitable to keep these classes together with the Rest clients, because they are closely related each other.
Auth config contains methods:
\begin{itemize}
    \item initialize() - it loads auth\_config.json configuration file and store the values from it by the clientId and clientSecret keys in the Map<String, String> (data container as similar as List),
    \item getClientId() - it returns a value from the Map container by the \textit{clientId} key and
    \item getClientSecrete() - it returns a value from the Map container by the \textit{clientSecret} key.
\end{itemize}
\textbf{Google API Key Reader} contains methods \textit{initialize()} and \textit{getApiKey()}.
\textit{initialize()} loads auth\_config.json configuration file and store the values from it by the clientId and clientSecret keys in the Map<String, String> (data container as similar as List).
\textit{getApiKey()} it returns a value from the Map container by the \textit{apiKey} key.
