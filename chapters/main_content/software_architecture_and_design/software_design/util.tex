\subsection{Util}\label{subsec:util}
This part contains utilities that facilitate work in the codebase.
It is separated into:
\begin{itemize}
    \item Error directory,
    \item Extensions directory,
    \item Preferences directory,
    \item Bloc Delegate class - it extends generic Bloc Delegate, and it logs all actions on every Bloc (on every Hydrated Bloc, too),
    \item Network Info class - it checks internet connection (it uses DataConnectionChecker class) and
    \item Validator class - it defines validation rules for e-mail address, password and full name.
\end{itemize}


\subsubsection{Error}\label{subsubsec:error}
This directory contains classes:
\begin{itemize}
    \item error\_handling - it contains a method that maps an error to action,
    \item Server Exception - it extends the general exception and
    \item Server Failure - it represents a returned fail message with the HTTP status code.
\end{itemize}


\subsubsection{Extensions}\label{subsubsec:extensions}
The Extension is a feature that can add functionality without the base class changes.
Besides, it helps the codebase facilitation and Extension can use even in cases where it is needed.
The Extensions directory contains:
\begin{itemize}
    \item Polygon Extension,
    \item Circle Extension,
    \item Number Extension and
    \item String Extension.
\end{itemize}
\textbf{Polygon Extension} offers to find out if a \textbf{point} consisted of latitude and longitude is lying in the given \textbf{Polygon}~\cite{googleMapsPolygon}.
It is based on the mathematical problem when we are looking for the result if y-axis of the given point goes through an even number of borderlines of the given Polygon.
To solve how to find out if the y-axis goes through is based on the problem which deals with two lines go through each other.
All the problem is described in this paper~\cite{geometricAlgorithms}.

\textbf{Circle Extension} offers to find out if a \textbf{point} consisted of latitude and longitude is lying in the given \textbf{Circle}~\cite{googleMapsCircles}.
It is based on the \textbf{Pythagorean theorem} that solves a computation of the longest line in a right triangle consisting of the spoken line and x and y-axis lines.

\textbf{Viewport Extension} contains \textit{isOutOf(Viewport viewport)} method, that extends the base interface of \textbf{LatLngBounds}~\cite{latLngBounds} class for the comparison if this viewport is out of the bounds of the given viewport.
The implementation is simple.
It compares all coordinates of these viewports each other.
If this viewport is out of the bounds of the given, it returns true.
Otherwise, it returns false.

\textbf{String Extension} contains methods for simplified work with Strings.
These methods are:
\begin{itemize}
    \item isValidEmail() - it checks if a given e-mail meets set Regular Expression,
    \item isValidPassword() - it checks if a given password meets set Regular Expression,
    \item isBlank() - it checks if a given string is empty or equal to an empty string and
    \item isValidFullName() - it checks if a given full name meets set Regular Expression.
\end{itemize}

\textbf{Number Extension} contains methods for the precision setting.
The methods are:
\begin{itemize}
    \item halfSize() - returns half of this number,
    \item toLatLonPrecision() - returns string of this number with precision on 6 decimal digits,
    \item toAltitudePrecision() - returns string of this number with precision on 4 decimal digits,
    \item addLatitudeDistance(double distance) - adds a given distance to this in latitude axis and
    \item addLongitudeDistance(double distance, double latitude) - adds a given distance to this in longitude axis.
\end{itemize}

\textbf{Date Extension} extends the base interface of the \textbf{Date} class and contains the get method \textit{suffix}, which returns a suffix in English for the number of the day of the given date.

\textbf{Translation Extension} extends the base interface of the \textbf{BuildContext} class and contains \textit{translate(String key)} method that mediates call of \textit{Flutteri18n.translate(key)} method with key parameter.
It returns a translation by phone setting properties, where the application is running, by the key in  \textbf{en.json} and \textbf{cs.json} file in \textbf{assets/i18n} directory.


\subsubsection{Preferences}\label{subsubsec:preferences}
This directory contains AppPreferences class that encapsulates Box class from Hive library, which provides NoSQL persistent data storage.
The Hive library has a simple interface for work with the nested Box class.
Further, it allows using Hive Annotations and define the objects for storing via an entity.
AppPreferences class contains public methods:
\begin{itemize}
    \item saveTokens(String apiToken, String refreshToken) - it calls \_putToken(String token) and \_putRefreshToken(String token) methods,
    \item destroyTokens() - it calls \_deleteToken() and \_deleteRefreshToken() methods,
    \item getToken() - it gets a value by API\_TOKEN constant,
    \item getRefreshToken() - it gets a value by API\_REFRESH\_TOKEN constant,
    \item isUserLoggedIn() - it checks if getToken() and getRefreshToken() return not empty string (If so, user is still logged in, otherwise logged out.),
    \item getRecentSearchResult() - it returns stored user search recent results and
    \item updateRecentSearchResult(SearchObject currentSearchedResult) - it updates stored user search recent results.
\end{itemize}
AppPreferences class contains private methods:
\begin{itemize}
    \item \_put(String key, value) - it puts a value by the given key into Box object storage,
    \item \_putToken(String token) - it calls \_put(String key, value) where the key is API\_TOKEN constant,
    \item \_putRefreshToken(String token) - it calls \_put(String key, value) where thekey is API\_REFRESH\_TOKEN constant,
    \item \_deleteToken() - it deletes a value by API\_TOKEN constant and
    \item \_deleteRefreshToken() - it deletes a value by API\_REFRESH\_TOKEN constant.
\end{itemize}
