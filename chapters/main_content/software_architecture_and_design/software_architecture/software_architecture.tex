\section{Software Architecture}\label{sec:software-architecture}
This section focuses on the architecture of this application implementation.
Because the definition of Software Architecture has been already clarified, it will be emphasized to certain components that were used in the implementation.

In Software Architecture has been proven a pattern of using libraries that offer generic classes that include already implemented logic that help to simplify and do code more readable thanks to the standards.
For better understanding, it will be described in the Bloc and Hydrated Bloc section~\ref{subsec:bloc-and-hydrated-bloc} of this chapter.

There would be useful to discuss the difference between common \textbf{Bloc} and its extended class \textbf{Hydrated Bloc}.
Both of them using the sink, and the stream concept is working on the stream subscription principle.
The Hydrated Bloc offers persistent storage storing and loading with \textit{fromJson} and \textit{toJson} methods.
It helps to have an application working more smoothly thank to storing of data in phone than alone continuous loading data via the network whenever a user needs them.


\subsection{Dependencies}\label{subsec:dependencies}
Flutter has a huge database of dependencies that are maintained by various developers.
The database is called \textbf{pub.dev}~\cite{pubDev} and offers to their users a summary for every dependency with the score analysis.
This analysis consists of Popularity, Health, Maintenance and final Overall.
The dependencies used in this mobile application are separated into following categories:
\begin{itemize}
    \item Icons,
    \item State management,
    \item NoSQL storage,
    \item Tools,
    \item Firebase,
    \item API,
    \item UI,
    \item Map and
    \item Generator.
\end{itemize}

\textbf{Icons} contains \textbf{cupertino\_icons} that provides iOS Cupertino Icons by Apple Inc.
\textbf{Material Design icons} are already part of the \textbf{Flutter SDK} library.

\textbf{State management} consists of \textbf{flutter\_bloc} and \textbf{hydrated\_bloc}.

\textbf{NoSQL storage} contains \textbf{hive} and \textbf{hive\_flutter}.
These dependecies ensure NoSQL storage that are for persitent data in offline mode and store things like JWT token.\cite{jwtToken}

\textbf{Tools} consist of:
\begin{itemize}
    \item \textbf{intl} - it ensures internationalization,
    \item \textbf{equatable} - it ensures that object extending it can be easily comparable by properties of that object,
    \item \textbf{flutter\_svg} - it ensures drawing SVG image as a widget,
    \item \textbf{flutter\_i18n} - it ensures Internationalization by the user phone settings,
    \item \textbf{path\_provider} - it ensures finding commonly used locations on the filesystem,
    \item \textbf{print\_lumberdash} - it ensures a convenient debug printing of an error message,
    \item \textbf{get\_it} - it allows using of Dependency Injection in code - it supports IoC principle,\cite{iocPrinciple}
    \item \textbf{dartz} - it allows using functional programming in Dart,
    \item \textbf{url\_launcher} - it ensures launching a URL in the mobile platform,
\end{itemize}

\textbf{Firebase} contains \textbf{firebase\_core} and \textbf{firebase\_messaging}.
These dependecies ensure connecting to Firebase Cloud and allowing push notifications.

\textbf{API} consists of:
\begin{itemize}
    \item \textbf{http} - it ensures making HTTP requests,
    \item \textbf{json\_annotation} - it ensures definition of annotations used by JSON serialization and deserialization,
    \item \textbf{data\_connection\_checker} - it checks for an internet connection by opening a socket,
    \item \textbf{retrofit} - it ensures type conversion \textbf{dio} client generator to create HTTP requests,
    \item \textbf{pretty\_dio\_logger} - it ensures a convenient debug printing of HTTP requests and responses,
    \item \textbf{dio} - it offers support of Interceptors, Global configuration, FormData, Request Cancellation, File downloading, Timeout etc.,
    \item \textbf{flutter\_screenutil} - it ensures adapting screen and font size.
\end{itemize}

\textbf{UI} consists of:
\begin{itemize}
    \item \textbf{sliding\_up\_panel} - it offers using done Sliding up panel,
    \item \textbf{animations} - it contains pre-canned animations for commonly-desired effects,
    \item \textbf{flutter\_sfsymbols} - it supports all iOS and Android devices.
\end{itemize}

\textbf{Map} consists of:
\begin{itemize}
    \item \textbf{google\_maps\_flutter} - it provides a Google Maps widget,
    \item \textbf{location} - it handles getting location on Android and iOS,
    \item \textbf{google\_maps\_webservice} - it offers using Google Web Service API that including Geocoding, Places, Directions, and so on.
\end{itemize}

\textbf{Generator} consists of:
\begin{itemize}
    \item \textbf{hive\_generator} - it generates storable object by Hive annotations,
    \item \textbf{retrofit\_generator} - it generates implementation of HTTP requests due to set annotations,
    \item \textbf{bloc\_test} - it generates testing blocs for unit testing,
    \item \textbf{mockito} - it generates Mocks for unit testing,
    \item \textbf{pedantic} - it ensures a linter that looking after to meet best practices,
    \item \textbf{build\_runner} - it provides a concrete way of generating files using Dart code,
    \item \textbf{json\_serializable} - it generates serializable object by JSON annotation.
\end{itemize}


\subsection{Bloc and Hydrated Bloc}\label{subsec:bloc-and-hydrated-bloc}
In the beginning, we counted on all of BLoCs will be the classic one.
After short time, we have found out the storable concept is more convenient for this purpose.
So, some following we changed to descendants of HydratedBloc.
We have created a list of the following Bloc classes:
\begin{itemize}
    \item User Profile Bloc,
    \item Aircraft Operation Bloc,
    \item Device Operation Bloc,
    \item Flight Operation Bloc,
    \item TODO
\end{itemize}

At the beginning, \textbf{User Profile Bloc} had extended from common Bloc but after increasing data traffic we decided to change it to Hydrated Bloc.
The reason is simple.
User properties do not change so often.
Due to it better to store the data in phone and update them when it changes.
User Profile bloc ensures:
\begin{itemize}
    \item fetching user data,
    \item log in user,
    \item sign up user,
    \item saving JWT token,
    \item
\end{itemize}

\textbf{Aircraft Operation Bloc}
%TODO: add description
Bloc ensures:
\begin{itemize}
    \item
\end{itemize}

\textbf{Device Operation Bloc}
%TODO: add description
Bloc ensures:
\begin{itemize}
    \item
\end{itemize}

\textbf{Flight OperationBloc}
%TODO: add description
Bloc ensures:
\begin{itemize}
    \item
\end{itemize}


\subsection{JsonSerializable classes}\label{subsec:jsonserializable-classes}
Json Serializable interface offers generating of  methods for serialization and deserialization classes that are annotated by Json Annotation.
These classes are converted and created the methods via the \textbf{build runner} library.
The advantage is a developer must not code these methods by hand, and he can re-generated every time it is needed to change the class structure.

\textbf{Request classes}
%TODO

\textbf{Response classes}
%TODO


\subsection{Dio class}\label{subsec:dio-class}
Due to need of using Interception and fill prepared HTTP requests with Authentication Bearer we defined our a custom \textbf{Dronetag Dio} class.
It initializes \textbf{Pretty Dio Logger} to make clear debug printing and Token Interceptor that extends \textbf{Interceptors Wrapper}.


\subsection{JsonSerializable}\label{subsec:jsonserializable-classes}
\textbf{Json Serializable} interface offers generating methods for serialization and deserialization classes.
\textbf{Json Annotation} annotates these classes.
These classes are converted, and the process creates the methods via the \textbf{build runner} library.
The advantage is that a developer must not code these methods by hand.
He can re-generated every time it is needed to change the class structure.

\textbf{Request classes} are used for store and serialization of data before the Rest client will send it.
This directory contains only \textbf{Login Request} and \textbf{Register Request} classes.

\textbf{Response classes} are used for store and deserialization of data after the Rest client has received it.
This directory contains:
\begin{itemize}
    \item Aircraft Detail Response - is used for passing aircraft data in Aircraft Repository,
    \item Aircraft Overview Response - is used for a nested object in Flight Info Response in Flight Repository,
    \item Api Error - is used for processing of API error message in each of Repositories,
    \item Device Detail Response - is used for passing device data in Device Repository,
    \item Device Live Response - is used for passing device live data in Device Repository,
    \item Device Telemetry Response - is used for passing device telemetry live data in Device Repository,
    \item Dronetag Overview Response - is used for a nested object in Flight Info Response in Flight Repository,
    \item Flight Info Response - is used for passing flight data in Flight Repository,
    \item Flight Statistics Response - is used for flight statistics data in Flight Repository,
    \item Location - is used for passing latitude and longitude data in almost each of Repositories,
    \item Region Response - is used for a nested object in Zone Response and Flight Info Response,
    \item Take Off Response - is used for a nested object in Flight Info Response in Flight Repository,
    \item Telemetry Response - is used for passing telemetry data in Device Repository,
    \item User Profile Response - is used for passing user data in User Repository,
    \item Zone Response - is used for passing zone data in Zone Repository.
\end{itemize}


\subsection{Dio class}\label{subsec:dio-class}
Due to the need to use Interception and fill prepared HTTP requests with Authentication Bearer, we defined our custom \textbf{Dronetag Dio} class.
It initializes \textbf{Pretty Dio Logger} to make a clear debug printing and \textbf{Token Interceptor} that extends \textbf{Interceptors Wrapper}.

