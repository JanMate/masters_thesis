\section{Software Architecture}\label{sec:software-architecture}
This section focuses on the architecture of this application implementation.
Because the definition of software architecture has been already clarified, it will be emphasized to certain components that were used in the implementation.

In software architecture, there has been verified a pattern of using libraries that offers using of generic classes.
These classes include already implemented logic that is based on standards, and thus it helps to simplify development and maintain code more readable.
For better understanding, Bloc and Hydrated Bloc classes will be described in the section~\ref{subsec:bloc-and-hydrated-bloc} of this chapter.
\newpage
There would be useful to discuss the difference between common Bloc and its extended class Hydrated Bloc.
Both of them use the sink and stream concept that works on the stream subscription principle.
The Hydrated Bloc offers persistent storage storing and loading with \textbf{fromJson} and \textbf{toJson} methods.
It helps to have an application working more smoothly due to storing of data in the phone.
Only continuous loading data via the network whenever a user needs them has significant impact to the response time in the application.


\subsection{Dependencies}\label{subsec:dependencies}
Flutter has a vast database of dependencies that are maintained by various developers.
The database is called pub.dev~\cite{pubDev} and offers a summary of every dependency on the score analysis to their users.
This analysis consists of popularity, health, maintenance and final overall.

The advantage of these dependencies is in the convenient way of using.
Developers can place their definition into \textbf{pubspec.yaml} file what is the dependency configuration file in the root directory of every Flutter project.
The definition consists of name and specific version.

The dependencies that are used in this mobile application are separated into the following categories:
\begin{itemize}
    \item Icons,
    \item State management,
    \item NoSQL storage,
    \item Tools,
    \item Firebase,
    \item API,
    \item UI,
    \item Map,
    \item Generator.
\end{itemize}

\textbf{Icons} contain \textbf{cupertino\_icons} library that provides iOS Cupertino Icons by Apple Inc.
Material Design icons are already part of the Flutter SDK library.

\textbf{State management} consists of \textbf{flutter\_bloc} and \textbf{hydrated\_bloc}.
These dependencies contain the Bloc and Hydrated Bloc implementation.
The implementation is described in the Bloc section~\ref{sec:bloc} of the Flutter Analysis chapter.
The usage of the implementation is described in the Bloc and Hydrated Bloc section~\ref{subsec:bloc-and-hydrated-bloc} of this chapter.

\textbf{NoSQL storage} contains \textbf{hive} and \textbf{hive\_flutter}.
These dependencies ensure NoSQL storage that is for persistent data in the offline mode and store things like JWT token.~\cite{jwtToken}
\newline
\newline
\textbf{Tools} consist of:
\begin{itemize}
    \item \textbf{intl} - it ensures internationalization,
    \item \textbf{equatable} - it ensures that object extending it can be easily comparable by properties of that object,
    \item \textbf{flutter\_svg} - it ensures drawing SVG image as a widget,
    \item \textbf{flutter\_i18n} - it ensures internationalization by the user phone settings,
    \item \textbf{path\_provider} - it ensures finding commonly used locations on the filesystem,
    \item \textbf{print\_lumberdash} - it ensures a convenient debug printing of an error message,
    \item \textbf{get\_it} - it allows using of Dependency Injection in code - it supports IoC principle,~\cite{iocPrinciple}
    \item \textbf{dartz} - it allows using functional programming in Dart,
    \item \textbf{url\_launcher} - it ensures launching a URL in the mobile platform,
\end{itemize}
\textbf{Firebase} contains \textbf{firebase\_core} and \textbf{firebase\_messaging}.
These dependecies ensure connecting to Firebase Cloud and allowing push notifications.
\newline
\textbf{API} consists of:
\begin{itemize}
    \item \textbf{http} - it ensures making HTTP requests,
    \item \textbf{json\_annotation} - it ensures the definition of annotations used by JSON serialization and deserialization,
    \item \textbf{data\_connection\_checker} - it checks for an internet connection by opening a socket,
    \item \textbf{retrofit} - it ensures type conversion dio client generator to create HTTP requests,
    \item \textbf{pretty\_dio\_logger} - it ensures a convenient debug printing of HTTP requests and responses,
    \item \textbf{dio} - it offers support of Interceptors, Global configuration, FormData, Request Cancellation, File downloading, Timeout, etc.,
    \item \textbf{flutter\_screenutil} - it ensures adapting screen and font size.
\end{itemize}
\textbf{UI} consists of:
\begin{itemize}
    \item \textbf{sliding\_up\_panel} - it offers using done Sliding up panel,
    \item \textbf{animations} - it contains pre-canned animations for commonly-desired effects,
    \item \textbf{flutter\_sfsymbols} - it supports all iOS and Android devices.
\end{itemize}
\textbf{Map} consists of:
\begin{itemize}
    \item \textbf{google\_maps\_flutter} - it provides a Google Maps widget,
    \item \textbf{location} - it handles getting a location on Android and iOS,
    \item \textbf{google\_maps\_webservice} - it offers using Google Web Service API that including Geocoding, Places, Directions, and so on.
\end{itemize}
\textbf{Generator} consists of:
\begin{itemize}
    \item \textbf{hive\_generator} - it generates storable object by Hive annotations,
    \item \textbf{retrofit\_generator} - it generates an implementation of HTTP requests due to set annotations,
    \item \textbf{bloc\_test} - it generates testing blocs for unit testing,
    \item \textbf{mockito} - it generates Mocks for unit testing,
    \item \textbf{pedantic} - it ensures a linter that is looking after to meet best practices,
    \item \textbf{build\_runner} - it provides a concrete way of generating files using Dart code,
    \item \textbf{json\_serializable} - it generates a serializable object by JSON annotation.
\end{itemize}


\subsection{Bloc and Hydrated Bloc}\label{subsec:bloc-and-hydrated-bloc}
In the beginning, we counted on all of BLoCs will be the classic one.
After a short time, we have found out that the storable concept is more convenient for this purpose.
So, some following, we changed to descendants of HydratedBloc.
We have created a list of the following Bloc classes:
\begin{itemize}
    \item User Profile Bloc,
    \item Authentication Bloc,
    \item Aircraft Operation Bloc,
    \item Device Operation Bloc,
    \item Flight Operation Bloc,
    \item Device Detail Bloc,
    \item Location Bloc,
    \item Map Bloc,
    \item Map Zone Bloc,
    \item Map Drone Bloc and
    \item Map Place Bloc.
\end{itemize}

In the beginning, \textbf{User Profile Bloc} had extended common Bloc, but after increasing data traffic, we decided to change it to Hydrated Bloc.
The reason is simple.
User properties do not change so often.
Due to it better to store the data on the phone and update them when it changes.
User Profile bloc ensures loading and updating user data.

\textbf{Authentication Bloc} extends Bloc.
Authentication Bloc ensures:
\begin{itemize}
    \item logging in a user,
    \item logging out user and
    \item finding out if a user is logged in.
\end{itemize}

\textbf{Aircraft Operation Bloc} extends Hydrated Bloc.
Aircraft Operation Bloc ensures:
\begin{itemize}
    \item loading aircraft data,
    \item adding new aircraft,
    \item updating an aircraft and
    \item deleting an aircraft.
\end{itemize}

\textbf{Device Operation Bloc} extends Hydrated Bloc.
Device Operation Bloc ensures:
\begin{itemize}
    \item loading device data,
    \item registration of a device to a user,
    \item updating a device and
    \item deleting a device.
\end{itemize}

\textbf{Flight Operation Bloc} extends Hydrated Bloc.
Flight Operation Bloc ensures:
\begin{itemize}
    \item loading flight statistics data,
    \item loading flights data,
    \item loading flight data,
    \item updating a flight and
    \item deleting a flight.
\end{itemize}

\textbf{Device Detail Bloc} extends Bloc.
Device Detail Bloc ensures:
\begin{itemize}
    \item switching Battery icon,
    \item switching LTE icon and
    \item switching Bluetooth icon.
\end{itemize}

\textbf{Location Bloc} extends Hydrated Bloc.
Location Bloc ensures focusing to a user location by GPS coordinates in the map in Dashboard.

\textbf{Map Bloc} extends Bloc.
Map Bloc is the most complex component in th whole project so far.
It looks after all user interaction with the map in the Dashboard.
Map Bloc mediates:
\begin{itemize}
    \item loading zones,
    \item finding all zones covering a given point,
    \item showing and closing zone selection,
    \item showing and closing zone detail,
    \item loading flying drones,
    \item showing and closing drone detail,
    \item focusing on a drone location,
    \item showing and closing place detail,
    \item focusing to a place location,
    \item pinning and unpinning place marker into the map and
    \item storing map object like Markers, Circles, Polygons and Polylines.
\end{itemize}
In addition, it has nested Blocs that listen the State changes.
These Blocs are the following:
\begin{itemize}
    \item Map Zone Bloc,
    \item Map Drone Bloc and
    \item Map Place Bloc.
\end{itemize}
These Blocs react on the Map Bloc State changes, which means when a new State is created, a nested Bloc catches it and generate its new Event.
The Event ensures the real action on the level of the nested Bloc thanks to mapping.

\textbf{Map Zone Bloc} extends Bloc.
Map Zone Bloc ensures:
\begin{itemize}
    \item loading zones,
    \item finding all zones covering a given point,
    \item showing and closing zone selection and
    \item showing and closing zone detail.
\end{itemize}

\textbf{Map Drone Bloc} extends Bloc.
Map Drone Bloc ensures:
\begin{itemize}
    \item loading flying drones,
    \item showing and closing drone detail and
    \item focusing on a drone location.
\end{itemize}

\textbf{Map Place Bloc} extends Bloc.
Map Place Bloc ensures:
\begin{itemize}
    \item showing and closing place detail,
    \item focusing on a place location and
    \item pinning and unpinning place marker into the map.
\end{itemize}


\subsection{JsonSerializable}\label{subsec:jsonserializable-classes}
Json Serializable interface offers generating methods for serialization and deserialization classes.
Json Annotation annotates these classes.
These classes are converted, and the process creates the methods via the build runner library.
The advantage is that a developer must not code these methods by hand.
He can re-generated every time it is needed to change the class structure.

\textbf{Request classes} are used for store and serialization of data before the Rest client will send it.
This directory contains only Login Request and Register Request classes.

\textbf{Response classes} are used for store and deserialization of data after the Rest client has received it.
This directory contains:
\begin{itemize}
    \item \textbf{Aircraft Detail Response} - is used for passing aircraft data in Aircraft Repository,
    \item \textbf{Aircraft Overview Response} - is used for a nested object in Flight Info Response in Flight Repository,
    \item \textbf{Api Error} - is used for processing of \acrshort{api} error message in each of Repositories,
    \item \textbf{Device Detail Response} - is used for passing device data in Device Repository,
    \item \textbf{Device Live Response} - is used for passing device live data in Device Repository,
    \item \textbf{Device Telemetry Response} - is used for passing device telemetry live data in Device Repository,
    \item \textbf{Dronetag Overview Response} - is used for a nested object in Flight Info Response in Flight Repository,
    \item \textbf{Flight Info Response} - is used for passing flight data in Flight Repository,
    \item \textbf{Flight Statistics Response} - is used for flight statistics data in Flight Repository,
    \item \textbf{Location} - is used for passing latitude and longitude data in almost each of Repositories,
    \item \textbf{Region Response} - is used for a nested object in Zone Response and Flight Info Response,
    \item \textbf{Take Off Response} - is used for a nested object in Flight Info Response in Flight Repository,
    \item \textbf{Telemetry Response} - is used for passing telemetry data in Device Repository,
    \item \textbf{User Profile Response} - is used for passing user data in User Repository,
    \item \textbf{Zone Response} - is used for passing zone data in Zone Repository.
\end{itemize}


\subsection{Dio}\label{subsec:dio-class}
Due to the need to use Interception and fill prepared \acrshort{http} requests with Authentication Bearer, it was defined the custom Dronetag Dio class.
It initializes Pretty Dio Logger to make a clear debug printing and Token Interceptor that extends Interceptors Wrapper.

