\section{Kubernetes}\label{sec:kubernetes}
"Kubernetes (K8s) is an open-source system for automating deployment, scaling, and management of containerized applications."~\cite{kubernetes}
It means that Kubernetes is a tool allowing a running cluster whose single nodes may be as Docker containers.
This approach ensures scalability, whereas the number of Docker containers can be huge and can be created and disposed of dynamically based on the loading.
The advantage of Kubernetes is that it can detect a note out of order and initialize and deploy a new one to stay the optimal performance

"It groups containers that make up an application into logical units for easy management and discovery.
Kubernetes builds upon \textit{15 years of experience of running production workloads at Google}~\cite{kubernetesArticle}, combined with best-of-breed ideas and practices from the community."~\cite{kubernetes}
Thanks to these experiences, Kubernetes is a fine grain, and the cluster hierarchy is merely maintainable.

"Though widespread interest in software containers is a relatively recent phenomenon, at Google we have been managing Linux containers at scale for more than ten years and built three different container-management systems in that time."~\cite{kubernetesArticle}
How you can see, the idea of containerization has a rich history.
However, there were Linux Containers before the Docker ones.
That is the reason why we use it for web development.
It is easy to scale, maintain, and able to deploy to Google Cloud Platform.