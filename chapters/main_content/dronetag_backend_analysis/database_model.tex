\section{Database Model}\label{sec:database-model}
There is a database model to store data in Dronetag Backend.
The database model consists of followings entities:
\begin{itemize}
    \item User,
    \item Aircraft,
    \item Device,
    \item Flight,
    \item Telemetry measurement,
    \item Organization,
    \item Fleet,
    \item Aircraft vendor,
    \item Aircraft model,
    \item Airspace zone,
    \item User preference.
\end{itemize}

\subsection{User}\label{subsec:user}
This entity represents a user who signs up and logs in to the application.
It consists of user credentials and identification information.
Attributes of this entity are the following:
\begin{itemize}
    \item \textbf{E-mail} - represents the e-mail that the user will be log in to the system and receive notifications,
    \item \textbf{Full name} - represents optional full name of the user,
    \item \textbf{Password hash} - represents a hashed password,
    \item \textbf{Phone number} - represents a phone number that the user will be able to contact in case of emergency,
    \item \textbf{Country} - represents a country where the user is used to fly.
\end{itemize}

\subsection{Aircraft}\label{subsec:aircraft}
This entity represents an aircraft that is connected with a Dronetag device.
Attributes of this entity are the following:
\begin{itemize}
    \item \textbf{Name} - represents the aircraft name for easier recognition in My aircraft list,
    \item \textbf{\acrshort{uas} Operator ID} - represents a unique code identifying a pilot who registered this aircraft in the \acrshort{uas}~(\acrlong{uas})~\cite{uas},
    \item \textbf{Weight} - represents the weight of the aircraft.
\end{itemize}

\subsection{Device}\label{subsec:device}
This entity represents a physical device that sends live information to the Dronetag platform.
Attributes of this entity are the following:
\begin{itemize}
    \item \textbf{Serial number} - represents a serial number of the device,
    \item \textbf{Name} - represents the device name for easier recognition in My aircraft list,
    \item \textbf{Type} - represents the model type of the device,
    \item \textbf{Last battery} - represents the last battery value in Volts,
    \item \textbf{Last \acrshort{rsrp}} - represents the \acrshort{rsrp}~(\acrlong{rsrp}) value.
\end{itemize}

\subsection{Flight}\label{subsec:flight}
This entity represents it represents a flight that a user has created.
Attributes of this entity are the following:
\begin{itemize}
    \item \textbf{Date planned start} - represents a start date of planned the flight,
    \item \textbf{Date planned finish} - represents a finish data of planned the flight,
    \item \textbf{Date started} - represents a real start date of the flight,
    \item \textbf{Date finished} - represents a real finish date of the flight,
    \item \textbf{Status} - represents a flight status - values can be planned, current, finished and canceled,
    \item \textbf{Distance} - represents a distance of the flight,
    \item \textbf{Duration} - represents a duration of the flight,
    \item \textbf{Region \acrshort{geojson}} - represents a reservation region in \acrshort{geojson} format~\cite{geoJson},
    \item \textbf{Max flight altitude} - represents a maximum flight altitude,
    \item \textbf{Takeoff latitude} - represents a latitude of a taking off position,
    \item \textbf{Takeoff longitude} - represents a longitude of a taking off position,
    \item \textbf{Takeoff Geo altitude} - represents a geological altitude of a taking off position,
    \item \textbf{Takeoff pressure} - represents a taking off pressure,
    \item \textbf{Public} - represents a boolean flag if the flight is public (visible for everyone).
\end{itemize}

\subsection{Telemetry Measurement}\label{subsec:telemetry-measurement}
This entity represents a telemetry measurement that the system receives and thanks so that a flight trajectory can be drawn.
Attributes of this entity are the following:
\begin{itemize}
    \item \textbf{Time} - represents a date with the time of measurement,
    \item \textbf{Latitude} - represents a measured latitude of a flying drone,
    \item \textbf{Longitude} - represents a measured longitude of a flying drone,
    \item \textbf{Altitude} - represents a measured altitude of a flying drone,
    \item \textbf{Geo altitude} - represents a measured geological altitude of a flying drone,
    \item \textbf{Velocity X} - represents a measured velocity in X-axis,
    \item \textbf{Velocity Y} - represents a measured velocity in Y-axis,
    \item \textbf{Velocity Z} - represents a measured velocity in Z-axis.
\end{itemize}

\subsection{Organization}\label{subsec:organization}
This entity represents an Organization for the fleet management functionality.
Attributes of this entity are the following:
\begin{itemize}
    \item \textbf{Name} - represents the name of the organization,
    \item \textbf{Description} - represents a text description of the organization.
\end{itemize}

\subsection{Fleet}\label{subsec:fleet}
This entity represents an organization's fleet management properties.
Attributes of this entity are the following:
\begin{itemize}
    \item \textbf{Name} - represents a name of the fleet,
    \item \textbf{Color} - represents a shown color of the fleet,
    \item \textbf{Deleted} - represents a boolean flag if the fleet was deleted.
\end{itemize}

\subsection{Aircraft Vendor}\label{subsec:aircraft-vendor}
This entity represents an aircraft vendor who manufactures drones.
It contains only the one attribute name that represents the vendor name.

\subsection{Aircraft Model}\label{subsec:aircraft-model}
This entity represents an aircraft model that belongs to a vendor.
Attributes of this entity are the following:
\begin{itemize}
    \item \textbf{Name} - represents the model name,
    \item \textbf{Weight} - represents the weight of the model,
    \item \textbf{Vendor ID} - represents a relationship to a Vendor.
\end{itemize}

\subsection{Airspace Zone}\label{subsec:airspace-zone}
This entity represents an airspace zone.
Attributes of this entity are the following:
\begin{itemize}
    \item \textbf{Name} - represents a name of the zone,
    \item \textbf{Country} - represents a country where the zone belongs,
    \item \textbf{Region \acrshort{geojson}} - represents a region in \acrshort{geojson} format~\cite{geoJson}.
\end{itemize}

\subsection{User Preference}\label{subsec:user-preference}
This entity represents a user preference that is needed to store and share among various client applications.
Attributes of this entity are the following:
\begin{itemize}
    \item \textbf{Property} - represents an identification of the property,
    \item \textbf{Value} - represents the preference value.
\end{itemize}

\subsection{Live Service Database Model}\label{subsec:live-service-database-model}
During the development, it had been found out the current Backend is not sufficient for Dronetag needs.
So it was decided to divide the backend model into the Backend and Live Service model.
The Live Service contains only live real-time temporary data, so a Redis database was deployed.

"Redis is an open source (BSD licensed), in-memory data structure store, used as a database, cache and message broker.
It supports data structures such as strings, hashes, lists, sets, sorted sets with range queries, bitmaps, hyperloglogs, geospatial indexes with radius queries and streams.
Redis has built-in replication, Lua scripting, LRU eviction, transactions and different levels of on-disk persistence, and provides high availability via Redis Sentinel and automatic partitioning with Redis Cluster."~\cite{redis}
It means that the Redis is real-time storage that persists data only for a short time.
That is the reason why it is suitable for this purpose.
The Live Service database model consists of a Device and Telemetry entity.

