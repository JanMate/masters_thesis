\section{Usability testing}\label{sec:usability-testing}

This section talks about usability testing what is a phase in the User interface design cycle.
This phase involves testing with a user who gets instruction by a scenario, and the task is to do it in his natural way.
The primary purpose of this testing is the detection of harmful design elements, and the determination of the arrangement mistakes.
The elements can be arranged in a screen, part of a screen, menu or drop-down menu.
It depends on the purpose of the concrete used application.

At first, we designed the Hi-Fi prototype by the concurrent applications, and we tried to learn from their design mistakes.
We had to decide if it is better to place a sliding panel with advisories on Dashboard or an animated button for a flight planning, how should look the profile menu screen and its icon or if it is useful to pin a place into the map.
After that, we have organized the usability testing with ordinary people who could be our potential customers.

During the testing, we were detecting a few crucial mistakes in the design.
They are accurately described in the following part of this section.

\subsection{Results}\label{subsec:results}
This part contains feedback from usability testing.
We have attended two usability testing.
The first one was at the beginning of mobile application development before we started implementing the essential application functions.
The other one was before the release of the alpha version of this application.
The following paragraphs are related to the other usability testing.

We have tested with six users.
One of them was from the Faculty of transportation sciences of Czech Technical University.
The others were colleagues from the Faculty of information technology of the same university.
Many of them were bachelor students, so it means they are between 20 and 25 of age.
The Faculty of transportation sciences student gave us an insight into what would expect a user with flight knowledge.
He describes how the Dashboard map should behave and how to understand various map layers and flight levels.
My Faculty’s colleagues were in the role of typical users and helped us to understand what they expect of the used elements in the screens.
For the briefness, we chose three of them.
Due to GDPR, we anonymized their name.

Person One:
\begin{itemize}
    \item "I would like to see separated groups in the Search screen with Headline text to distinguish what the search thing is."
    \item "I like the Full drone detail screen.
    It contains everything that I want to know. The map is impressive."
\end{itemize}
Person Two:
\begin{itemize}
    \item "It would be nice to have a list of pinned places on the Dashboard map.
    If I pin many places, I will get lost between them."
    \item "I like the Country list selection with the Search bar.
    It is good-looking and useful."
    \item "The application is stable in the Alpha version.
    I like the simplicity of the Dashboard screen."
\end{itemize}
Person Three:
\begin{itemize}
    \item "The place pin in the Dashboard map is too small, and I can see it."
    \item "The zone colors with opacity are too light for sunlight.
    I would not have to see them outside."
    \item "I like the tiles in My Aircrafts and My Devices screens.
    The swiping for delete looks excellent.
    However, I am missing an option to select these objects and their actions (For example, delete, share, or change order)."
\end{itemize}


\subsection{Main mistakes in UI design}\label{subsec:main-mistakes-in-ui-design}
During the usability testing, we discovered the critical findings that are described in this part.
We expected some of them.
Some of them surprised us and forced us to consider a change in the user interface.
For example, we found out that:
\begin{enumerate}
    \item It is not as intuitive as the users expect the searching drones and devices from the Search button.
    The user expects only searching places, whereas it is a part of the Dashboard, where the map is the essential element.
    \item It is not sure the bottom panel is possible to pull up.
    So, it is one of the reasons we decided to omit the bottom sliding up panel with advisories so far.
    The other is that it is not known and approved the final legislation so far, and it is still in progress.
    \item The assigning of the drone to the device is not useful.
    It is better to allow a user to set a default device and default drone, and allow the user to change them when he is planning a flight.
    \item A user appreciates an option to move with the map in the In-flight screen.
\end{enumerate}
This knowledge has had a significant impact on changing screens to reach a better user’s attention.
