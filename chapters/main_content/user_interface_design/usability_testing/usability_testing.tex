\section{Usability testing}\label{sec:usability-testing}
In the User interface design cycle is the usability testing phase.
This phase involves testing with user who gets instruction by a scenario, and the task is to do it in his natural way.
The main purpose of this testing is detection of bad design elements and determination of the arrangement mistakes.
The elements can be arranged in a screen, part of a screen, menu or drop-down menu.
It depends on the purpose of concrete used application.

At first, we designed the Hi-fi prototype by the concurrent applications and we tried to learn from their design mistakes.
We had to decide if it is better to place a sliding panel with advisories on Dashboard or an animated button for a flight planning, how should look the profile menu screen and its icon or if it is useful to pin a place into the map.
After that we have organized the usability testing with common people who could be our potential customers.

During the testing we were detecting a few essential mistakes in the design.
They are closely described in following subsection Results \ref{subsec:results}.

\subsection{Results}\label{subsec:results}
We have tested with 6 users.
One of them was from Faculty of transportation sciencies, the others were my colleagues from Faculty of information technology.
Many of them were bachelor students, so it means they are between 20 and 25 of age.
The Faculty of transportation sciences' student gave me an insight what would expect a user with flight knowledge.
He describes me how should behave the map on the Dashboard and how I understand various map layers and flight levels.
My Faculty's colleagues were in the role of common users and helped me to understand what they expect of the used elements in the screens.

\subsection{Main mistakes in UI design}\label{subsec:main-mistakes-in-ui-design}
