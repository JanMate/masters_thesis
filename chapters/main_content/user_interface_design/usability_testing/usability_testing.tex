\section{Usability Testing}\label{sec:usability-testing}

This section talks about Usability Testing what is a phase in the User Interface Design cycle.
This phase involves testing with users who get instruction by a scenario, and the task that they get they should do in their natural way.
The primary purpose of this testing is the detection of harmful design elements, and the determination of the arrangement mistakes.
The elements can be arranged in a screen, part of a screen, menu or drop-down menu.
It depends on the purpose of the concrete application.

At first, the Hi-Fi prototype was designed and inspired by the competitor applications.
In the design was effort to learn from design mistakes of the competitor applications and solve their flaws.
It had to be decided if it is better to place a sliding panel with advisories on Dashboard or an animated button for a flight planning, how should look the profile screen and its icon, or if it is useful to pin a place into the map.
After that, it has been organized the usability testing with ordinary people who could be Dronetag's potential customers.

During the testing, it was being detected a few crucial mistakes in the design.
They are accurately described in the following part of this section.

\subsection{Results}\label{subsec:results}
This part contains feedback from usability testing.
It has been attended two usability testing.
The first was at the beginning of mobile application development before the implementing of the essential application functions was started.
The other was before the release of the alpha version of this application.
The following paragraphs relate to the usability testing before the alpha version.

It has been tested with six users.
One of them was from the Faculty of transportation sciences of Czech Technical University in Prague.
The others were colleagues from the Faculty of information technology of the same university.
Many of them were bachelor students, so it means they are between 20 and 25 of age.
The student of the Faculty of transportation sciences gave an insight into what would expect users with flight knowledge.
He describes how the Dashboard map should behave and how to understand various map layers and flight levels.
My Faculty’s colleagues were in the role of typical users and helped to understand what they expect of the used elements in the screens.
For the briefness, there were chosen and cited three of them.
Due to GDPR, their names were anonymized.
\newline
\newline
Person One:
\begin{itemize}
    \item "I would like to see separated groups in the Search screen with Headline text to distinguish what the searched thing is."
    \item "I like the Full drone detail screen.
    It contains everything that I want to know.
    The map is impressive."
\end{itemize}
Person Two:
\begin{itemize}
    \item "It would be nice to have a list of pinned places on the Dashboard map.
    If I pin many places, I will get lost between them."
    \item "I like the Country list selection with the Search bar in the Profile screen.
    It is good-looking and useful."
    \item "The application is stable in the Alpha version.
    I like the simplicity of the Dashboard screen."
\end{itemize}
Person Three:
\begin{itemize}
    \item "The place pin in the Dashboard map is too small, and I cannot see it."
    \item "The zone colors with opacity are too light for sunlight.
    I would not have to see them outside."
    \item "I like the tiles in My Aircrafts and My Devices screens.
    The swiping for delete looks excellent.
    However, I am missing an option to select these objects and their actions (For example, delete, share, or change order)."
\end{itemize}


\subsection{Main Mistakes in UI Design}\label{subsec:main-mistakes-in-ui-design}
During the usability testing, there were discovered the critical findings that are described in this part.
Some of them were expected.
Some of them were not expected and forced to change the future goals in the user interface design.
For example, the following flaws were found:
\begin{enumerate}
    \item It is not so intuitive to expect the searching drones and devices from the Search button in the Dashboard.
    The users expect only searching places, whereas it is a part of the Dashboard, where the map is the essential element.
    \item It is not obvious that the bottom panel is possible to pull up.
    So, it is the first of the reasons to decide to omit the bottom sliding up panel with advisories.
    The other is that it is not known and approved the final legislation so far, and it is still in progress.
    \item The assigning of the drone to the device is not useful.
    It is better to allow users to set a default device and default drone, and allow the users to change them when they are planning a flight.
    \item Users appreciate an option to move with the map in the In-flight screen.
\end{enumerate}
This knowledge has had a significant impact on changing screens to reach a better attention of users.
