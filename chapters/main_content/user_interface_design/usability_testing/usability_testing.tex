\section{Usability testing}\label{sec:usability-testing}
This section talks about usability testing what is phase in the User interface design cycle.
This phase involves testing with user who gets instruction by a scenario, and the task is to do it in his natural way.
The main purpose of this testing is detection of bad design elements and determination of the arrangement mistakes.
The elements can be arranged in a screen, part of a screen, menu or drop-down menu.
It depends on the purpose of concrete used application.

At first, we designed the Hi-Fi prototype by the concurrent applications and we tried to learn from their design mistakes.
We had to decide if it is better to place a sliding panel with advisories on Dashboard or an animated button for a flight planning, how should look the profile menu screen and its icon or if it is useful to pin a place into the map.
After that we have organized the usability testing with common people who could be our potential customers.

During the testing we were detecting a few essential mistakes in the design.
They are closely described in following subsection Results \ref{subsec:results}.

\subsection{Results}\label{subsec:results}
We have tested with 6 users.
One of them was from Faculty of transportation sciencies, the others were my colleagues from Faculty of information technology.
Many of them were bachelor students, so it means they are between 20 and 25 of age.
The Faculty of transportation sciences' student gave me an insight what would expect a user with flight knowledge.
He describes me how should behave the map on the Dashboard and how I understand various map layers and flight levels.
My Faculty's colleagues were in the role of common users and helped me to understand what they expect of the used elements in the screens.
%TODO: add feed back to prototype and real application

\subsection{Main mistakes in UI design}\label{subsec:main-mistakes-in-ui-design}
During the usability testing we discovered the interesting findings that are described in this section.
We expected some of them.
Some of them surprised us and were forcing to consider a change in User interface.
For example, we found out that:
\begin{enumerate} %Todo: translate
    \item it is not as so much intuitive as the users expect the searching drones and divices from Search button.
    The user expects only searching places, whereas it is a part of Dashboard where the map si the key element.
    \item it is not sure the bottom panel is possible pull up.
    So, it is one of the reasons, why we decided to omit the bottom sliding up panel with advisories so far.
    Ten druhý je, že pro zatím není známa a schválená legislativa a neustále se formuje.
    \item přiřazení konkrétního dronu k zařízení není příliš užitečně.
    Vhodnější je umožnit uživateli nastavit si default device a default drone a při plánování letu mu jej dovolit změnit.
    \item uživatel uvítá možnost hýbat s mapou v In-flight modu.
\end{enumerate}
Tyto poznatky měly velký dopad na to, že jsme nějaké screeny změnili tak, aby byly blížší vnímání uživatele.