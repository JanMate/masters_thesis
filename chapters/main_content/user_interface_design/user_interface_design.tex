\chapter{User Interface Design}\label{ch:user-interface-design}
This chapter is focused on the user interface design of the Dronetag mobile application.
It starts with an explanation of difference between UI (User Interface) and UX (User Experience), and continues with a description about the user interface structure of this application.
So, it contains a list of screens that are shown to users when the users go through the application.

The main difference between UI and UX is in the approach.
UI approach emphasizes a combination of visual elements such that the final concept of visualization fits together.
UX approach emphasizes an arrangement of these visual elements such that their arrangement was logical, and it was close to human thinking.~\cite{prototyping}
There are many visual elements.
They can be:
\begin{itemize}
    \item A text label,
    \item A text field,
    \item A button,
    \item A checkbox,
    \item A List view,
    \item A menu,
    \item An icon,
    \item An image,
    \item And others.
\end{itemize}
UI approach deals with their skin~-~for example, a color, size, shadow, border, shape, and others.
UX approach focuses on the size and position on the screen.
The arrangement of elements in screens should be comprehensive and should make sense to users.

The user interface design phase usually includes Lo-Fi (Low Fidelity) and Hi-Fi (High Fidelity) prototyping.~\cite{effectivePrototyping}
There will be clarified the difference between them.
Lo-Fi, how the name hints, is a simplified sketch without colors with a purpose to incorporate the elements on a screen together.
It is cheaper than Hi-Fi and takes only a few hours to finish.
The benefits are:
\begin{itemize}
    \item Focus on design and concepts,
    \item Focus on design and concepts,
    \item Accessible to everyone.~\cite{hiFiLoFiPrototypeArticle}
\end{itemize}

In opposite, Hi-Fi, how the name hints, includes a fidelity design that should correspond to the final product.
It is more expensive than Lo-Fi and usually takes a few days.
The benefits are:
\begin{itemize}
    \item More familiar to users,
    \item Pinpoint specific components to test,
    \item More presentable to stakeholders.~\cite{hiFiLoFiPrototypeArticle}
\end{itemize}
To more information about fidelity, anyone can read this article~\cite{hiFiLoFiPrototypeArticle}.

Lo-Fi prototype has not been created because Dronetag is Start-up, a company with a limited budget, so there is only the Hi-Fi prototype.
When Hi-Fi was being made, it was inspired by a few competitor applications, which show permitted flight zones and danger areas.
These applications are described in the Related projects chapter~\ref{ch:related-projects}, so they will not be discussed anymore.


\section{Hi-Fi Prototype}\label{sec:hi-fi-prototype}
This section consists of screens in the application with a detailed description of their elements.
The \acrshort{hifi} prototype was made by Marian Hlav{\' a}{\v c} in Adobe XD~\cite{adobeXD} and is available online~\cite{hifiPrototype}.
It was emphasized to the simplicity and briefness and kept the focus on the essential \acrshort{ui} and \acrshort{ux} design rules.
These rules are based on "Jakob Nielsen's 10 general principles for interaction design".~\cite{nnGroup}
It describes key aspects of User Interface Design.
The heuristics by Jakob Nielsen are the following:
\begin{enumerate}
    \item Visibility of system status,
    \item Match between system and the real world,
    \item User control and freedom,
    \item Consistency and standards,
    \item Error prevention,
    \item Recognition rather than recall,
    \item Flexibility and efficiency of use,
    \item Aesthetic and minimalist design,
    \item Help users recognize, diagnose, and recover from errors,
    \item Help and documentation.
\end{enumerate}

Applications built on these heuristics will be successful and useful because these heuristics are based on various psychological researches and usability testing with ordinary users.

\subsection{Dashboard}\label{subsec:dashboard2}

The \textbf{Dashboard screen} is immediately after the \textbf{Splash screen}.
We is counted on the fact a user will spend most of the time on the \textbf{Dashboard screen}, and so it contains the of the functionalities.
This screen is divided into:
\begin{itemize}
    \item Top panel,
    \item Map controls and
    \item Fly now button.
\end{itemize}
These elements are in the higher stack layer below the map where are placed flying drones, place pins and restricted areas.
Besides, it allows displaying without wasteful \textbf{POIs} (Point of Interest)\cite{poi}, satellite map or standard map, which contains POIs, precisely like \textbf{Google Maps} application.

The Top panel is consists of:
\begin{itemize}
    \item Device status - it contains current information about default device,
    \item Search button - it is for searching of places, devices, aircrafts and zones and
    \item Profile button - it represents the main menu of the whole application.
\end{itemize}
If a user has already logged in and planned a flight, Profile button contains a light blue circle with the number of planned flights in the top right on the circuit of that circle button.
The key difference we can see on the picture~\ref{fig:dashboard}.

Map controls are consists of:
\begin{itemize}
    \item My location button - it redirects to his position due to his location by GPS coordinates,
    \item Map layers button - it is an offer allowing us to choose a map style and
    \item Fly now button - this button disappears options Fly now and Plan a flight for already logged in user and Log in and sign up button for an unauthorized user.
\end{itemize}
These are additional elements showing on the \textbf{Dashboard screen}:
\begin{itemize}
    \item Drone detail panel, (Figure \ref{fig:dashboard_drone_detail})
    \item Place detail panel,
    \item Zone detail panel, (Figure \ref{fig:dashboard_zone_detail})
    \item Zone selection panel, and
    \item Map layer panel.
\end{itemize}

The \textbf{Drone detail panel} is for showing necessary information and a flying drone and its flight parameters.
Besides, it contains the button to redirection to the full drone detail screen, where a user can see all information about live flight immediately.

The \textbf{Place detail panel} contains necessary information about the place.
Also, if a user wants to plan a flight from this place, it allows us to pin this place until the user unpins it from the map and Plan a flight button.
Place pinning into the map is demonstrated by a color change of the pin and color change of the icon in the panel.

The \textbf{Zone detail panel} contains necessary information about the given zone, such as a lower altitude level, upper altitude level, name of the zone, and validation from and zone status.

The \textbf{Zone selection} is used to select a zone if a user clicks to a place in the map where is an intersection of more zones.
Simultaneously, when this selection is shown, the selected zones are highlighted.
If a user chooses a zone, the zone will show Zone detail panel and will highlight only one particular zone instead of the intersection.

We focused on the primary purpose of informing a user about drones and restricted areas around him.
So, we decided to keep a minimalistic design because the user needs to emphasize what is important to him.


\begin{figure}
    \centering
    \begin{minipage}{.4\textwidth}
        \centering
        \includegraphics[width=.7\linewidth]{assets/user_interface_design/dashboard/dashboard.png}
        \caption{Dashboard}
        \label{fig:dashboard}
    \end{minipage}%
    \hspace{.05\linewidth}
    \begin{minipage}{.4\textwidth}
        \centering
        \includegraphics[width=.7\linewidth]{assets/user_interface_design/dashboard/dashboard_drone_detail.png}
        \caption{Dashboard, Drone detail}
        \label{fig:dashboard_drone_detail}
    \end{minipage}
    \label{fig:dashboard_all}
\end{figure}


\subsection{Login and Registration screens}\label{subsec:login-screen}
\textbf{Login screen} contains buttons for log in and sign up to the system.
Login buttons allows to log in via email, Google account and it going to add log in via Apple account.
Sign up button allows to sign in only via email.

\textbf{Log in Screen} consists of two text fields.
The first is for typing an e-mail and the other is for typing a password.
After success log in, the user is redirected to \textbf{Dashboard screen}.

\textbf{Sign up screen} is consist of following three text fields:
\begin{itemize}
    \item the first is for typing an e-mail address through the user will log in to the system,
    \item the second is for typing a password and
    \item the last one is for typing a user name.
\end{itemize}
The last one is optional.
The user has to agree with terms of using before he finishes the registration.
After successful signing up, the user is logged in and redirected to \textbf{Dashboard screen}.


\subsection{Profile Screens}\label{subsec:profile-screens}
The Profile screen~(Figure~\ref{fig:profile}) represents the main menu of the application.
Users have granted access to their profile detail, My management container, and the users can see the set default aircraft and device.
Besides, if they belong to an organization, it shows them the organization name and the button to switch the fleet mode.
My management container contains My Flights, My Devices and My Aircrafts buttons.

The Profile detail screen contains the user properties like the full name, phone number and country.
Also, it allows changing the user's contact e-mail and password.


\begin{figure}
    \centering
    \begin{minipage}{.4\textwidth}
        \centering
        \includegraphics[width=.7\linewidth]{assets/user_interface_design/dashboard/dashboard_zone_detail.png}
        \caption{Zone detail}
        \label{fig:dashboard_zone_detail}
    \end{minipage}%
    \hspace{.05\linewidth}
    \begin{minipage}{.4\textwidth}
        \centering
        \includegraphics[width=.7\linewidth]{assets/user_interface_design/profile/profile.png}
        \caption{Profile}
        \label{fig:profile}
    \end{minipage}
    \label{fig:profile_all}
\end{figure}


\subsection{Device Screens}\label{subsec:device-screens}
The Devices screen~(Figure~\ref{fig:devices}) will be showed after users click on My Devices button in the Profile screen.
This screen contains a list of registered Dronetag devices and Add button.
The Add button redirects the users to the Device registration screen.

The Device screen contains all information about the given device, including the real--time data.
Users can set the device as default or set the device preferences by their own needs.

The Device registration screen is for registration of a new device that the users bought.
It contains two text fields for typing - a serial number, and an optional name for better identification.
After successful registration, the users are redirected to the Devices screen.


\subsection{Aircrafts screens}\label{subsec:aircrafts-screens}
%todo: add text


\begin{figure}
    \centering
    \begin{minipage}{.45\textwidth}
        \centering
        \includegraphics[width=.7\linewidth]{assets/user_interface_design/aircraft/aircrafts.png}
        \caption{[A24] Aircrafts}
        \label{fig:aircrafts}
    \end{minipage}%
    \hspace{.05\linewidth}
    \begin{minipage}{.45\textwidth}
        \centering
        \includegraphics[width=.7\linewidth]{assets/user_interface_design/aircraft/new_aircraft.png}
        \caption{[A25] New Aircraft}
        \label{fig:new_aircraft}
    \end{minipage}
    \label{fig:aircrafts_all}
\end{figure}


\subsection{Flight screens}\label{subsec:flight-screens}
\textbf{Flights screen} (Figure \ref{fig:flights}) is for displaying list of the flights.
This screen contains a flight statistics and allows finding certain flights by chosen filters.
By the every flight, it shows its state in case it is planned and current flight.
In addition, it contains the button for export the whole flight history.
If a user choose planned flight, it shows to him the Flight plan screen, what is planned flight summary - it consists of various parameters.
For finnished flight is possible to show \textbf{Flight detail} and for current flight \textbf{In-flight detail}.

\textbf{Flight detail screen} contains all information about a flight including the option to play flight track from start to the end.
\textbf{In-flight detail screen} contains information about the flight in real time.
Thanks to \textbf{sliding up panel} we can see more information and still check if a drone did not escape the reservation area.

\textbf{Plan a flight} is a group of three screens that represent a wizard for flight plan.
In the first screen, user sets identification properties and planned time of flight.
In the second screen, user sets a polygon for planned flight and its maximum altitude.
In the last screen, user confirms set parameters of planned flight.
It is like a summary.


\begin{figure}
    \centering
    \begin{minipage}{.45\textwidth}
        \centering
        \includegraphics[width=.7\linewidth]{assets/user_interface_design/device/devices.png}
        \caption{[A20] Devices}
        \label{fig:devices}
    \end{minipage}%
    \hspace{.05\linewidth}
    \begin{minipage}{.45\textwidth}
        \centering
        \includegraphics[width=.7\linewidth]{assets/user_interface_design/flight/flights.png}
        \caption{[A50] Flights}
        \label{fig:flights}
    \end{minipage}%
    \label{fig:flight_all}
\end{figure}


\subsection{Search screen}\label{subsec:search-screen}
\textbf{Opened Search screen} is the first what a user see when he clicks on the Search button in \textbf{Dashboard}.
It shows recent results that user has searched at the last time.
The result can be a drone, aircraft, place or zone.

If a user uses \textbf{Search text field} and type a text string, it will show to him a result with an icon for better identification.
If a user choose an item in the list of results, it redirects him to either to \textbf{Aircraft detail} or \textbf{Device detail}, or even a place in the map in \textbf{Dashboard}.
After choosing that item, it will appear in the list of \textbf{Recent} results.



\section{Usability testing}\label{sec:usability-testing}
This section talks about usability testing what is phase in the User interface design cycle.
This phase involves testing with user who gets instruction by a scenario, and the task is to do it in his natural way.
The main purpose of this testing is detection of bad design elements and determination of the arrangement mistakes.
The elements can be arranged in a screen, part of a screen, menu or drop-down menu.
It depends on the purpose of concrete used application.

At first, we designed the Hi-Fi prototype by the concurrent applications and we tried to learn from their design mistakes.
We had to decide if it is better to place a sliding panel with advisories on Dashboard or an animated button for a flight planning, how should look the profile menu screen and its icon or if it is useful to pin a place into the map.
After that we have organized the usability testing with common people who could be our potential customers.

During the testing we were detecting a few essential mistakes in the design.
They are closely described in following subsection Results \ref{subsec:results}.

\subsection{Results}\label{subsec:results}
We have tested with 6 users.
One of them was from Faculty of transportation sciencies, the others were my colleagues from Faculty of information technology.
Many of them were bachelor students, so it means they are between 20 and 25 of age.
The Faculty of transportation sciences' student gave me an insight what would expect a user with flight knowledge.
He describes me how should behave the map on the Dashboard and how I understand various map layers and flight levels.
My Faculty's colleagues were in the role of common users and helped me to understand what they expect of the used elements in the screens.
%TODO: add feed back to prototype and real application

\subsection{Main mistakes in UI design}\label{subsec:main-mistakes-in-ui-design}
During the usability testing we discovered the interesting findings that are described in this section.
We expected some of them.
Some of them surprised us and were forcing to consider a change in User interface.
For example, we found out that:
\begin{enumerate} %Todo: translate
    \item it is not as so much intuitive as the users expect the searching drones and divices from Search button.
    The user expects only searching places, whereas it is a part of Dashboard where the map si the key element.
    \item it is not sure the bottom panel is possible pull up.
    So, it is one of the reasons, why we decided to omit the bottom sliding up panel with advisories so far.
    Ten druhý je, že pro zatím není známa a schválená legislativa a neustále se formuje.
    \item přiřazení konkrétního dronu k zařízení není příliš užitečně.
    Vhodnější je umožnit uživateli nastavit si default device a default drone a při plánování letu mu jej dovolit změnit.
    \item uživatel uvítá možnost hýbat s mapou v In-flight modu.
\end{enumerate}
Tyto poznatky měly velký dopad na to, že jsme nějaké screeny změnili tak, aby byly blížší vnímání uživatele.
