\section{Util}\label{sec:util}
%todo

\subsection{Error}\label{subsec:error}
%todo

\subsection{Extensions}\label{subsec:extensions}
The Extensions directory contains:
\begin{itemize}
    \item Polygon Extension,
    \item Circle Extension,
    \item Number Extension and
    \item String Extension.
\end{itemize}
\textbf{Polygon Extension} offers to find out if a \textbf{Point}~\cite{} consisted of latitude and longitude is lying in the given \textbf{Polygon}~\cite{}.
It is based on the mathematical problem when we looking for the result if a y-axis of a point goes through an even number of border lines of the given polygon.
To solve how find out if the y-axis goes through is based on the problem which deals with two lines go through each other.
All the problem is described in this paper~\cite{}.
%todo \cite{}: http://www.dcs.gla.ac.uk/~pat/52233/slides/Geometry1x1.pdf

\textbf{Circle Extension} offers to find out if a \textbf{Point}~\cite{} consisted of latitude and longitude is lying in the given \textbf{Circle}~\cite{}.
It is based on the \textbf{Pythagorian theorem} that solves a computation of the longest line in a .

\textbf{String Extension}

\textbf{Number Extension}
Only notes:
A component with interesting logic could be the Polygon Extension.
It is for finding out if a given point lies in the Polygon.

%todo
And computation of distance and transform it on lat long coordinates !

\subsection{Preferences}\label{subsec:preferences}
%todo
