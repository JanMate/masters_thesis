zajištění bezpečnosti provozu dronů.
V té době to byly spíše hračky, které si hobby piloti sestavovali sami doma.
Proto je provázeli obvykle technické problémy.
Po určité době v nich začali vidět potenciál velké firmy a začali na nich stavět svou vizi budoucnosti.
Například společnost Amazon představila svůj koncept autonomních dronů schopných doručovat zboží z logistických skladů přímo k zákazníkům domů.
Zvýšením počtů majitelů dronů začali vznikat problémy.
Většina hobby pilotů při létech porušovala zákon a ohrožovala bezpečnost leteckého provozu.
Důvodem byla neznalost patřičných zákonů, které do té doby museli znát pouze piloty posádkových letounů.
Proto se postupem času začal tento problém řešit.
Přestože se na piloty bezpilotních letounů částečně vztahují i ty posádkových letounů.
Na druhou stranu, pilotům dronů stačí znát značení zón  a nemusí být zatížení kompletní instruktáží jako jsou piloti posádkových letounů.
Pro bezpečný provoz je nutné znát základní typy a značení zón.
Tyto zóny se dělí na několik typů.
Tyto hladiny (levels) se dělí do několika skupin podle altitude uváděných v feet units.
Dále je nutné si uvědomit, že uvedená výška může být dvojího typu.
To znamená, že si musíte na tyto jednotky dát pozor a a v případě AMSL si uvědomit v jaké aktuální výšce se nacházíte, aby jste při letu nevletěli do zakázané zóny.
Dále se vzdušný prostor dělí na třídy A-G.
Pro třídy A-E platí nutnost svůj let hlásit a Air Navigation Service of the Czech Republic letům přiřazuje povolení.
Třídy F a G jsou v nižšší letové hladině a není v nich nutné žádat o povolení Air Navigation Service of the Czech Republic.
Motivací podílet se na této práci pro mě byla v tom, že se mohu realizovat ve vývoji mobilních applikací.
Důvodem byl můj zájem si vyzkoušet něco nového, jelikož mám zatím pouze zkušenosti z webového vývoje.
Dalším důvodem byly úspěchy v soutěží dvou zakládajících členů společnosti Dronetag
ale dokonce se umístili v soutěžích na evropské a světové úrovni
Motivací pro mě bylo také možnost podílet se na produktu, který má velký potenciál a můžeme ho nazvat produktem budoucnosti.
Aplikace bude splňovat všechny funkční požadavky.
vykreslování zón v mapě, zobrazování létajících dronů opatřených Dronetag zařízením v reálném čase, plánování letů a management vlastních letounů, zařízení a historie letů.
