V práci jsme si řekli o problému s porušováním zákonu letání s dronem a také jsme si popsali typy zón, kterým se musí piloti dronů vyhýbat a letové hladiny.
Navíc jsme si popsali rozdíly mezi těmito typy.

Věnovali jsme se analýze současných řešení pro podporu pilotů bezpilotních letounů.
V této analýze jsme si popsali klíčové rozdíly.

Kapitola o backend infrastructure jsme si představili celý stack technologií poskytující data, datový model a propojení těchto technologií.

Nezapomněli jsme ani na základy Flutteru a důvody, proč jsme si pro vývoj cross-platform mobile application vybrali právě Flutter a stručné srovnání konkurenčních cross-platform frameworků.

Poté jsme se věnovali navržené struktuře codu mobilní application implementation.
Dále se průběžně zjištuje, že návrh UX/UI nebyl zpočátku zcela bezchybný.

K tomuto zjištění jsem také došli v průběhu user testing.
Proto dochází k postupnému upravování a uskupování prvku podle výsledků průběžného uživatelského testování.
Popsali jsme si jak je realizovaná implementační část práce včetně nezbytných kroků, které jsou nutné k zprovoznění vývojového prostředí.

Popis obsahuje také konfigurační soubory, které se z bezpečnostních důvodů nesmí verzovat pomocí verzovacích nástrojů.

Řekli jsme si také o zajištění kvality softwaru a jaké metody jsme v mobilní aplikaci využili.

Popsali jsme si výhody a využití automatizovaných testů a vybraných funkcionalit otestovaných Unit testy.
Jak jsem již zmínil jedním z produktů společnosti Dronetag je mobilní applikace, která bude zajišťovat bezpečný provoz dronů.

Tato applikace slouží pouze jako klient, a proto je přímo závislá na celé platformě, zejména na backendu a Live Service.
Proto byl vývoj mobilní applikace byl rozdělen na 2 hlavní fáze nazvané jako Alpha and Beta.
Pro vydání první verze takzvané Alpha, jsme stanovili na 31.5.2020.

Tato verze obsahuje základní fukcionalitu jako je sledování zón, management dronů, devices a historii letů včetně vyhledávání a také plánování letů ve dvou režimech.
Ten první je Fly now, což umožní uživateli rychlé spuštění zaznamenávání letu.

Tou druhou je Plan a flight, což umožní uživateli naplánovat let na určitý čas a na určeném místě.
tedy bude poskytovat hlavní funkcionalitu aplikace.
V dalších několika málo měsíců budeme připravovat druhou verzi tak nazvanou Beta.

Tato verze už bude klást důraz na optimalizaci výkonu a využívat část dat v offline módu.
Chceme tedy zajistit plnou funkcionalitu aplikace i v místech se špatným mobilním signálem.
Navíc se počítá s celkovou realizací funkcionality fleet managementu pro Organizace.
Tato funkcionalita by byla vhodná pro využití v průmyslu a zajišťovala by bezpečnost větší skupiny dronů ve vyhrazeném prostoru.

Hlavní skupinu zákazníků by mohly tvořit společnosti podnikající v delivery of goods (roznášení zboží), sledování okolí letišť and medical system to transfer medical materials to inaccessible places.

se tedy zaměří na optimalizaci a vylepšení všeho, co se v první verzi neosvědčí.

Na závěr je nutné zdůraznit, že struktura mobilní aplikace se stále mění a rozrůstá.
Je to dáno tím, že se neustále implementují další a další funkcionality, které přispějí k širšímu využití aplikace.
Navíc se, se stále větším a větším využítím dronů v průmyslu, bude nutné držet krok s dobou a tyto technologie neustále vylepšovat a přizpůsobovat k potřebám zákazníků.

Dronetag na tyto požadavky reaguje a snaží se nabídnout produkt, který to poskytuje a bude celosvětově využitelný.