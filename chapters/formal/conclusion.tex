\setsecnumdepth{part}
\chapter{Conclusion}\label{ch:conclusion}
% repeat problem statement
In this thesis we talked about the problem of breaking laws by flight drones and we also described zones type which the pilots have to avoid and flight levels.
%V práci jsme si řekli o problému s porušováním zákonu letání s dronem a také jsme si popsali typy zón, kterým se musí piloti dronů vyhýbat a letové hladiny.
In addition, we clarified the differences among these zone types. %Navíc jsme si popsali rozdíly mezi těmito typy.

%related projects
We emphasized to the analysis of related solutions for drone pilot support. %Věnovali jsme se analýze současných řešení pro podporu pilotů bezpilotních letounů.
In this analysis we described the key differences. %V této analýze jsme si popsali klíčové rozdíly.

%backend infrastructure
In the backend infrastructure chapter we introduced the all technology stack providing data, data model and connection of this technologies. %Kapitola o backend infrastructure jsme si představili celý stack technologií poskytující data, datový model a propojení těchto technologií.

%popsali jsme si basics of Flutter
We did not forget eighter on the Flutter basics and reasons why we chose the Flutter for cross-platform mobile application development and brief comparison of the competition cross-platform frameworks.
%Nezapomněli jsme ani na základy Flutteru a důvody, proč jsme si pro vývoj cross-platform mobile application vybrali právě Flutter a stručné srovnání konkurenčních cross-platform frameworků.

%software architecture and design
We described the difference between Software architecture and design.
Then, we emphasized to the designed code structure of mobile application implementation.
%Poté jsme se věnovali navržené struktuře codu mobilní application implementation.

%UI design
Further, it is still finding out that the UX/UI design has not been flawless at the beginning. %Dále se průběžně zjištuje, že návrh UX/UI nebyl zpočátku zcela bezchybný.
This finding, we have revealed during the usability testing. %K tomuto zjištění jsem také došli v průběhu user testing.
Hence, it is still improving, changing and grouping of the elements by the results from the first version mobile application usability testing. %Proto dochází k postupnému upravování a uskupování prvku podle výsledků průběžného uživatelského testování.

%deployment and testing
We described how to realize the implementation part of this thesis including neccesary steps to run the project in the development environment.
%Popsali jsme si jak je realizovaná implementační část práce včetně nezbytných kroků, které jsou nutné k zprovoznění vývojového prostředí.
This description also contain the configuration files that must not check in with the version systems for security reasons.
%Popis obsahuje také konfigurační soubory, které se z bezpečnostních důvodů nesmí verzovat pomocí verzovacích nástrojů.

We also talked about Software quality assurance and which methods we used in the mobile application. %Řekli jsme si také o zajištění kvality softwaru a jaké metody jsme v mobilní aplikaci využili.
We clarified the advantages and use cases of automation tests and chosen functions tested by Unit tests. %Popsali jsme si výhody a využití automatizovaných testů a vybraných funkcionalit otestovaných Unit testy.

%evaluation
How I mentioned, one of the Dronetag products is the mobile application that will ensure safer drone operations. %Jak jsem již zmínil jedním z produktů společnosti Dronetag je mobilní applikace, která bude zajišťovat bezpečný provoz dronů.
This application is used for only client thus it is directly dependent on the all platform, especially on the backend and Live service. %Tato applikace slouží pouze jako klient, a proto je přímo závislá na celé platformě, zejména na backendu a Live Service.
Hence, the development was divided into two main phases called Alpha and Beta. %Proto byl vývoj mobilní applikace byl rozdělen na 2 hlavní fáze nazvané jako Alpha and Beta.

For the Alpha version release we set the date to May 31st, 2020. %Pro vydání první verze takzvané Alpha, jsme stanovili na 31.5.2020.
This version contains the basic functionality like watching zones, drone, device and flight management including searching and flight planning in two modes.
%Tato verze obsahuje základní fukcionalitu jako je sledování zón, management dronů, devices a historii letů včetně vyhledávání a také plánování letů ve dvou režimech.
The first one is Fly now that allows a user to quickly record his flight. %Ten první je Fly now, což umožní uživateli rychlé spuštění zaznamenávání letu.
The other one is Plan a flight that allows a user to plan a flight on the certain time and place. %Tou druhou je Plan a flight, což umožní uživateli naplánovat let na určitý čas a na určeném místě.
So, the Alpha version will provide the main application functionality. %tedy bude poskytovat hlavní funkcionalitu aplikace.

In next many months, we will be preparing the second version called Beta. %V dalších několika málo měsíců budeme připravovat druhou verzi tak nazvanou Beta.
This version will emphasizes to the performance optimalization and use some data in offline mode. %Tato verze už bude klást důraz na optimalizaci výkonu a využívat část dat v offline módu.
So, we wants ensure the full application functionality even in places with out of service. %Chceme tedy zajistit plnou funkcionalitu aplikace i v místech se špatným mobilním signálem.
In addition, it counts on the complete functionality realization of the fleet management for Organizations. %Navíc se počítá s celkovou realizací funkcionality fleet managementu pro Organizace.
This functionality would be suitable for the industry purposes and ensure the safety for bigger group of drones in the reserved area.
%Tato funkcionalita by byla vhodná pro využití v průmyslu a zajišťovala by bezpečnost větší skupiny dronů ve vyhrazeném prostoru.
The most of the customers could be the companies doing business in delivery of goods, watching around of the airports and medical system to transfer medical materials to inaccessible places.
%Hlavní skupinu zákazníků by mohly tvořit společnosti podnikající v delivery of goods (roznášení zboží), sledování okolí letišť and medical system to transfer medical materials to inaccessible places.
So, the Beta version will focus on the optimalization and improvement of anything what will not have prove in the Alpha version.  %se tedy zaměří na optimalizaci a vylepšení všeho, co se v první verzi neosvědčí.

%final conclusion
In conclusion, it is needed to emphasize the fact that the mobile application structure is still growing. %Na závěr je nutné zdůraznit, že struktura mobilní aplikace se stále mění a rozrůstá.
It is given by the fact there are still implemented new and new functions that will contribute to wider using of the application. %Je to dáno tím, že se neustále implementují další a další funkcionality, které přispějí k širšímu využití aplikace.
In addition, there will be needed to be up-to-date with still bigger and bigger using of drones in industry and still improving and addapting these technologies on the customers' requirements.
%Navíc se, se stále větším a větším využítím dronů v průmyslu, bude nutné držet krok s dobou a tyto technologie neustále vylepšovat a přizpůsobovat k potřebám zákazníků.
Dronetag company reacts on these requirements a tries to offer a product which provides it and can be world-widely usable.
%Dronetag na tyto požadavky reaguje a snaží se nabídnout produkt, který to poskytuje a bude celosvětově využitelný.
