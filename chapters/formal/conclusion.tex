\setsecnumdepth{part}
\chapter{Conclusion}\label{ch:conclusion}
% repeat problem statement
In this thesis, we talked about the problem of breaking laws by flight drones, and we also described zones type which the pilots have to avoid and flight levels.
Also, we clarified the differences among these zone types.

%related projects
We emphasized the analysis of related solutions for drone pilot support.
In this analysis, we described the key differences.

%backend infrastructure
In the backend infrastructure chapter, we introduced the all technology stack providing data, data model, and connection of these technologies.

%Flutter Analysis
We did not forget eighter on the Flutter basics and reasons why we chose the Flutter for cross-platform mobile application development and a brief comparison of the competition cross-platform frameworks.

%software architecture and design
We described the difference between Software architecture and design.
Then, we emphasized to the designed code structure of mobile application implementation.

%UI design
Further, it is still finding out that the UX/UI design has not been flawless initially.
This finding, we have revealed during the usability testing.
Hence, it is still improving, changing, and grouping the elements using the results from the first version of mobile application usability testing.

%deployment and testing
We described how to realize the implementation part of this thesis includes necessary steps to run the project in the development environment.
This description also contains the configuration files that must not check in with the version systems for security reasons.

We also talked about Software quality assurance and which methods we used in the mobile application.
We clarified the advantages and use cases of automation tests and chosen functions tested by Unit tests.

%evaluation
How we mentioned, one of the Dronetag products is the mobile application that will ensure safer drone operations.
This application is used as a client.
Thus it is directly dependent on the all platform, especially on the backend and Live service.
Hence, the development was divided into two main phases, called Alpha and Beta.

For the Alpha version release, we set the date to May 31st, 2020.
This version contains essential functionality like watching zones, drones, devices, and flight management, including searching and flight planning in two modes.
The first one is Fly now that allows a user to record his flight quickly.
The other is Plan a flight that allows a user to plan a flight on a specific time and place.
So, the Alpha version will provide the main application functionality.

In the next many months, we will be preparing the second version called Beta.
This version will emphasize performance optimization and use some data in an offline mode.
So, we want to ensure the full application functionality even in places with out-of-service.
Besides, it counts on the complete functionality realization of the fleet management for Organizations.
This functionality would be suitable for industrial purposes and ensure the safety of a bigger group of drones in the reserved area.
Most of the customers could be the companies doing business in the delivery of goods, watching around the airports, and medical systems to transfer medical materials to inaccessible places.
So, the Beta version will focus on the optimization and improvement of anything that will not have proven in the Alpha version.

%final conclusion
In conclusion, it is needed to emphasize the fact that the mobile application structure is still growing.
The fact gives it there are still implemented new and new functions that will contribute to more comprehensive using of the application.
Besides, there will need to be up-to-date with still more prominent use of drones in industry and still improving and adapting these technologies to the customers’ requirements.
The Dronetag company reacts to these requirements, so it tries to offer a product that provides it and can be world-widely usable.
