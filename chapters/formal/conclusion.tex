\setsecnumdepth{part}
\chapter{Conclusion}\label{ch:conclusion}
Write what I have done when I was working on this thesis and summarize the results.
Possible steps in the future.

%Todo: repeat problem statement
Popsali jsme si rozdíl mezi zónami, kterým se musí piloti dronů vyhýbat a letové hladiny.

%TOdo: popsali jsme si basics of Flutter
Nezapomněli jsme ani na základy Flutteru a důvody, proč jsme si pro vývoj cross-platform mobile application vybrali právě Flutter.


%todo: backend infrastructure
Kapitola o backed infrastructure jsme si představili celý stack technologií poskytující data, datový model a propojení těchto technologií.

%TODo: UI design
Na závěr je nutné zdůraznit, že struktura mobilní aplikace se stále mění a rozrůstá.
Je to dáno tím, že se neustále implementují další funkcionality, které přispějí k širšímu využití.
Dále se průběžně zjištuje, že návrh UX/UI nebyl zpočátku zcela bezchybný.
K tomuto zjištění jsem také došli v průběhu user testing.
Proto dochází k postupnému upravování a uskupování prvku podle výsledků průběžného uživatelského testování.

%TODO: implementation with configuration
Popsali jsme si jak je realizovaná implementační část práce včetně nezbytných kroků, které jsou nutné k zprovoznění vývojového prostředí.
Popis obsahuje také konfigurační soubory, které se z bezpečnostních důvodů nesmí verzovat pomocí verzovacích nástrojů.

%ToDo: testing
Řekli jsme si také o zajištění kvality softwaru a jaké metody jsme v mobilní aplikaci využili.
Popsali jsme si výhody a využití automatizovaných testů a vybraných funkcionalit otestovaných Unit testy a integračními testy.

%TOdO: evaluation
Vývoj mobilní applikace byl rozdělen na 2 hlavní fáze nazvané jako Alpha and Beta.
Alpha bude poskytovat hlavní funkcionalitu aplikace.
Beta se zaměří na optimalizaci a vylepšení všeho, co se v první verzi neosvědčí.
