\setsecnumdepth{part}
\chapter{Conclusion}\label{ch:conclusion}
% repeat problem statement
In this thesis, the problem of lawbreaking by flight drones was discussed.
Also, zone types which the pilots have to avoid were described, as well as the flight levels.
Besides, there were clarified the differences among these zone types.
%related projects
An analysis of related solutions for drone pilot support was conducted, and the key differences among them were described in the Related projects chapter~\ref{ch:related-projects}.
%backend infrastructure
There was introduced the technology stack which provides data, data modelling, and connectivity for these technologies in the Backend infrastructure chapter~\ref{ch:dronetag-web-infrastructure}.
%Flutter Analysis
The basics of Flutter and the reasons why it was chosen for cross-platform mobile application development were discussed, including a brief comparison of the competing cross-platform frameworks.
In addition, the Bloc pattern was clarified and compared with its extended class, Hydrated Bloc.

%software architecture and design
There was described the difference between software architecture and design.
Then, it was emphasized the designed code structure of mobile application implementation that was divided into the software architecture and design components, according to the concept appproach.
%UI design
Further, it was found that the \acrshort{ux}/\acrshort{ui} design initially had some flaws, which was revealed during the usability testing.
Hence, it is still improving and changing, based on the results from the usability testing of the first version of the mobile application.
%deployment and testing
There was described how to realize the implementation of this thesis, including the necessary steps to run the project in the development environment.
The implementation meets all functional requirements, specifically drawing flight zones on the map, showing real-time data about flying drones with a Dronetag device attached, and flight planning and management of aircrafts, devices and flight history.
The environment description also contains the configuration files which should not be versioned with the version systems for security reasons.
It was also discussed software quality assurance and which methods are used in the mobile application.
There were clarified the advantages and use cases of automated tests and chosen functions tested by Unit tests.

%evaluation
\newpage
There was mentioned that one of the Dronetag products is a mobile application that will ensure safer drone operations;
this application is used as a client.
Thus, it is directly dependent on the entire platform, especially on the Backend and Live Service.
Hence, the development was divided into two main phases, called Alpha and Beta.

For the Alpha release was set the date to May 31st, 2020.
This version contains essential functionality like watching zones, drones, devices, and flight management, including searching and flight planning in two modes.
The first mode is "Fly now," which allows users to record his flight quickly.
The other is "Plan a flight," which allows users to plan a flight for a specific time and place.
As such, the Alpha version will provide the main application functionality.

Over the coming months, there will being prepared the second version called Beta.
This version will emphasize performance optimization and use some data in an offline mode as there is required to ensure full application functionality even in places without service.
In addition, it includes the complete functionality realization of the fleet management for organizations.
This functionality would be suitable for industrial purposes and ensure the safety of a larger group of drones in the reserved area.
Most of the customers would be companies delivering goods, monitoring airports, and transferring medical materials to inaccessible places.
Overall, the Beta version will focus on the optimization and improvement of any problems that have arisen in the Alpha version.

%final conclusion
In conclusion, it is important to emphasize the fact that the mobile application structure is still growing.
This fact means there are still to be implemented new functions that will contribute to more widespread use of the application.
In addition, there will be a need to stay up-to-date with new developments in the use of drones in business, always improving and adapting these technologies to the customers' requirements.
The Dronetag company is reactive to these requirements, so it offers a product that satisfies them and can be used worldwide.
