\setsecnumdepth{part}
\chapter{Introduction}\label{ch:introduction}
This thesis focuses on safer drone operations.
In 2012, drones began to spread in the Czech Republic.
At that time, they were toys that were built by hobby pilots' hands at home.
Thus, there were technical problems with them.

After a while, the big companies began to see potential and then have started to build their future vision.
For example, Amazon introduced its autonomous drone concept to deliver goods from logistic warehouses directly to customers’ homes.
This news was published in Forbes magazine by Jillian D'Onfro.\cite{amazonArticle}
The customer could be more satisfied with this shopping because the time of delivery can be cut on as a tiny part of the time.
Even Amazon has decided to produce its drone products.
One of them is the Prime Air delivery drone.

To make the delivery process more comfortable, Amazon is developing an autonomous drone application to minimize costs.
Hence, we can expect a significant expansion of autonomous aircrafts.
Potentially, it could be possible to use drones to deliver medical materials in case of a significant accident.
Especially, after a hurricane, fight conflict, or even during epidemic diseases like Coronavirus this time.

With the increasing number of drone owners, it has started to arise problems.
Most of the hobby pilots broke the law and endangered the aviation safety when they were flying.
The reason was ignorance of laws that were needed to know for crew flight pilots until that time.
Hence, it has started to solve this problem.
Although the laws for crew flight pilots are partly related to the drone pilot, too.
On the other hand, the drone pilots know the symbols and restricted areas and do not have to be concentrated on the full course of instruction like crew flight pilots.


It is needed to know the primary type and description of zones for safer operation.
These zones separate into many types.
They are FIS (Flight Information Region), CTR (Control Zone), TMA (Terminal Maneuvering Area), ATZ (Aerodrome Traffic Zone), airports, P (Prohibited Area), R (Restricted Area), D (Dangerous Area), TRA (Temporary Reserved Area) and TSA (Temporary Segregated Area).\cite{airspace}

Besides, there are essential parameters in the zones.
Lower and upper altitude level.
These flight levels divide into some groups by the altitude given in feet units.
Further, it is needed to realize that the given altitude can be one of two types.
AMSL (Above Mean Sea Level) and AGL (Above Ground Level).
It means we have to beware of the type of given altitudes, and in the case of AMSL, we have to realize our current altitude to avoid restricted areas.


Also, the airspace divides into A-G classes.\cite{airspace}
There is an obligation to announce a flight for the A-E classes to authorities, and the Navigation Service of the Czech Republic approves the flight.
The F and G classes are lower flight levels, so it is not needed to request an authorization on the Air Navigation Service of the Czech Republic.

Air Navigation Service of the Czech Republic released a website L{\' e}tejte zodpov{\v e}dn{\v e} (Fly carefully) about drone problems and wrote advisories for hobby drone pilots in an understandable way.
This website is closely described in Related projects chapter~\ref{sec:fly-carefully}.


\section{Motivation}\label{sec:motivation}
There was a motivation to participate in this thesis because I wanted to realize myself in mobile application development.
The reason was that I am interested in trying something new, and I have only been experienced in web development so far.
The other reason was the successes in various competitions of both of the Dronetag company co-founders Luk{\' a}{\v s} Brchl and Marian Hlav{\' a}{\v c}.
They won a hackathon, and even they won in the competitions on the European and world level.
There was another motivation to participate in a product with huge potential, and we can call it a product of the future.
There is a reason why business with drones is called exponential.


\section{Objectives}\label{sec:objectives}
The goal of this thesis is to develop a cross-platform mobile application for safer drone operation.
The application will be implemented with the best-practice principles in the Flutter framework.
Concurrently, this thesis contains a description fo the basic Flutter framework concepts in the Flutter Analysis chapter.\ref{ch:flutter-analysis}
The implementation contains partly testing by unit tests and deployment specifications, too.
The application will meet all functional requirements.
These requirements are drawing of flight zones in the map, showing real-time data about flying drones attached to a Dronetag device, flight planning and management of aircrafts, devices and flight history.

