\setsecnumdepth{part}
\chapter{Introduction}\label{ch:introduction}
This thesis is focused on ... zajištění bezpečnosti provozu dronů.

%Since 2017, RLP is solving a problem with managing of flight drone operations.
%It is the reason why this thesis was established.

V roce 2012 se začali v ČR rozmáhat drony.
V té době to byly spíše hračky, které si hobby piloti sestavovali sami doma.
Proto je provázeli obvykle technické problémy.
Po určité době v nich začali vidět potenciál velské firmy a začali na nich stavět svou vizi.
Například společnost Amazon představila svůj koncept autonomních dronů schopných doručovat zboží z logistických skladů přímo k zákazníkům domů.

%TODO: todo
At first, I would like to talk about the solutions in the world especially using of drones in the industry.
For example, Amazon Inc., company is working on a proof of concept which can use drones to delivery a package ordered by a customer.
This news was published in Forbes magazine by Jillian D'Onfro.\cite{amazonArticle}
The customer could be more satisfied with this shopping because the time of delivering can be cut on as the tiny part of the time as now.
Even Amazon has decided to produce their own drone products.
One of them is Prime Air delivery drone.

To make the delivery process easier, Amazon is developing autonomous drones application to minimize costs.
Amazon using/usage/application

autonomous drones application

Potentially, it could be possible to use drones to delivery the medical materials in case of big accident.
Especially, after a hurricane, fight conflict, or even during epidemic diseases like Coronavirus.

Zvýšením počtů majitelů dronů začali vznikat problémy.
Většina hobby pilotů při létech porušovala zákon a ohrožovala bezpečnost leteckého provozu.
Důvodem byla neznalost patřičných zákonů, které do té doby museli znát pouze piloty posádkových letounů.
Proto se postupem času začal tento problém řešit.
Přestože se na piloty bezpilotních letounů vztahují i ty posádkových letounů.
Na druhou stranu, pilotům dronů stačí znát značení zón and restricted areas a nemusí být zatížení kompletní instruktáží jako jsou piloti posádkových letounů.

Pro bezpečný provoz je nutné znát základní typy a značení zón.
Tyto zóny se dělí na několik typů.
Jsou to CTR, TMA, airports, R, D, A.
Dále je u těchto zón důležitý parametr lower and upper altitude level.
Tyto hladiny (levels) je dělí do několika skupin.
Jsou to


%TODO
Drones problematics and advisories (rules)

%\cite{} http://www.laacr.cz/SiteCollectionDocuments/rozdeleni-vp/p03-%20nakolenik%202019%20vnejsek%20rez%20D2.jpg
Popsat letové hladiny a zóny


!!! Použito v abstraktu !!!
Tato práce se zabývá celým vývojem multiplatformní mobilní aplikace ve frameworku Flutter pro plánování a sledování letů.
Tento vývoj zahrnuje kompletní analýzu, návrh a samotnou implementaci.

V analýze je kladen důraz na rozbor existující webové platformy a celé infrastuktury, která poskytuje data klientským aplikacím a vystavuje API endpoint aplikacím třetích stran.

Návrh se dělí na architekturu a design mobilní aplikace z pohledu softwaru.
Zde je popsána celá struktura aplikace a využité architektonické patterny frameworku Flutter.

Implementace popisuje detaily o realizaci a obsahuje podrobný návod pro spuštění ve vývojovém, testovacím a produkčním prostředí.
Zároveň obsahuje popis důležitých konfiguračních souborů, které se liší napříč různými prostředími a mají bezpečnostní charakter.

\section{Motivation}\label{sec:motivation}
Motivací podílet se na této práci pro mě byla v tom, že se mohu realizovat ve vývoji mobilních applikací.
Dalším důvodem byly úspěchy v soutěží dvou zakládajících členů společnosti Dronetag Luk{\' a}{\v s} Brchl and Marian Hlav{\' a}{\v c}.
Nejenže vyhráli hackathon, ale dokonce se umístili v soutěžích na evropské a světové úrovni.
Motivící pro mě bylo také možnost podílet se na produktu budoucnosti, který bude mít v velké využití a už nyní má velký potenciál.
Má svůj důvod, proč se trhu s drony říká exponenciální.
%TODO
využití dronů v průmyslu + trochu historie
Exponenciální trh

úspěchy Dronetagu v soutěžích, Hackathonech


\section{Objectives}\label{sec:objectives}
%TODO
Cílem této práce je vyvinou cross-platform mobile application pro zajištění bezpečnosti dronů.
Aplikace bude implementovaná podle best-practice principů ve frameworku Flutter.
Zároveň tato práce obsahuje popis základních konceptů frameworku Flutter v kapitole Analysis of Flutter.
Implementace zahrnuje i částečné otestování unit testy a integrační testy a deployment specification.
Aplikace bude splňovat všechny funkční požadavky.
Těmi jsou vykreslování zón v mapě, zobrazování létajících dronů opatřených Dronetag zařízením v reálném čase, plánování letů a management vlastních letounů, zařízení a historie letů.

%\section{Problem statement}\label{sec:problem-statement}
%Drones using in industry increasing every year.
%Nowadays,
