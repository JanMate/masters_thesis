\setsecnumdepth{part}
\chapter{Introduction}\label{ch:introduction}
In 2012, drones began to spread around the world.
At that time, they were mainly toys built by hobby pilots in the home.
Thus, there were technical problems with them.

After a while, large companies began to see the commercial potential of drones and started to incorporate them into their visions for the future.
Amazon introduced its autonomous drone concept to transport goods from logistics warehouses directly to customers.~\cite{amazonArticle}
As customer satisfaction increases even with small improvements to delivery time, Amazon has decided to produce its drone products, including the Prime Air delivery drone.

To make the delivery process more convenient and minimize costs, Amazon is developing an autonomous drone application.
Hence, a significant increase in the number of autonomous aircrafts can be expected.
Potentially, drones could be used to deliver medical materials in case of a major accident or disasters such as a hurricane, military conflict, or even during a disease epidemic like the current novel coronavirus pandemic.

With the increasing number of drone owners, problems have started to arise.
Many hobby pilots have broken the law and endangered aviation safety when they were flying.
Ignorance of laws was common because there had been no need for drone pilots to know any of the rules of aviation that traditional pilots must know.
However, this problem has begun to be resolved.
Some laws for traditional pilots are applicable to the drone pilot.
Drone pilots need to know certain symbols and respect restricted areas, but they do not have to complete a full course of instruction like traditional pilots.
For example, it is necessary for drone pilots to know the primary types and descriptions of zones for safe operation.
These zones can be separated into different types based on their properties.

There are a variety of key parameters in the zones, for example, lower and upper altitude level.
These flight levels are further divided into groups based on the altitude in feet.
Furthermore, it is important to realize that the given altitude can be one of two types: \acrshort{amsl}~(\acrlong{amsl}) and \acrshort{agl}~(\acrlong{agl}).
This means it is necessary to be clear about the type of given altitudes, and in the case of AMSL, to be aware of the current altitude to avoid restricted areas.


Overall, airspace is divided into 7 classes, A--G.~\cite{airspace}
There is an obligation to announce a flight from through class A--E to the authorities, and the Air Navigation Service of the Czech Republic approves the flight.
The F and G classes are lower flight levels, so it is not necessary to request an authorization for them.

The Air Navigation Service of the Czech Republic hosts the website "L{\' e}tejte zodpov{\v e}dn{\v e} (Fly Carefully)" about drone problems and has written advisories for hobby drone pilots using clear, understandable language.
This website is described in detail in the Related projects chapter~\ref{sec:fly-carefully}.


\section{Motivation}\label{sec:motivation}
One motivation for choosing this thesis was to gain experience with mobile application development.
I was interested in trying something new, and I had previously only had experience with web development.
The other reason was the success in various competitions of both Dronetag's company co-founders, Luk{\' a}{\v s} Brchl and Marian Hlav{\' a}{\v c}.
They won the Space application hackathon 2018~\cite{spaceHackathon}, and they have won multiple competitions on the European and world levels.
Finally, I wanted to work with a product with a great deal of potential, and drones are considered a product of the future;
business with drones is expected to experience exponential growth.


\section{Objectives}\label{sec:objectives}
The goal of this thesis is to develop a cross-platform mobile application for safer drone operation.
The application will be implemented the best-practice principles using the Flutter framework.
In addition, this thesis contains a description of the basic Flutter framework concepts in the Flutter Analysis chapter~\ref{ch:flutter-analysis}.
The implementation contains both testing by unit tests and deployment specifications, too.
The application meets all functional requirements, specifically drawing flight zones on the map, showing real-time data about flying drones with a Dronetag device attached, and flight planning and management of aircrafts, devices and flight history.

