% arara: pdflatex
% arara: bibtex
% arara: pdflatex
% arara: pdflatex


% options:
% thesis=B bachelor's thesis
% thesis=M master's thesis
% czech thesis in Czech language
% english thesis in English language
% hidelinks remove colour boxes around hyperlinks

\documentclass[thesis=M,english]{template/FITthesis}[2019/12/23]

\usepackage[utf8]{inputenc} % LaTeX source encoded as UTF-8
% \usepackage[latin2]{inputenc} % LaTeX source encoded as ISO-8859-2
% \usepackage[cp1250]{inputenc} % LaTeX source encoded as Windows-1250

% \usepackage{subfig} %subfigures
% \usepackage{amsmath} %advanced maths
% \usepackage{amssymb} %additional math symbols

\usepackage{graphicx}
\usepackage{dirtree} %directory tree visualisation
\usepackage{pdfpages}
\usepackage{import}
% \usepackage{xevlna}

% % list of acronyms
% \usepackage[acronym,nonumberlist,toc,numberedsection=autolabel]{glossaries}
% \iflanguage{czech}{\renewcommand*{\acronymname}{Seznam pou{\v z}it{\' y}ch zkratek}}{}
% \makeglossaries

% % % % % % % % % % % % % % % % % % % % % % % % % % % % % % 
% EDIT THIS
% % % % % % % % % % % % % % % % % % % % % % % % % % % % % % 

\department{Department of Software Engineering}
\title{Cross-platform mobile application for safer drone operations}
\authorGN{Jan} %author's given name/names
\authorFN{Mat{\v e}jka} %author's surname
\author{Jan Mat{\v e}jka} %author's name without academic degrees
\authorWithDegrees{Bc. Jan Mat{\v e}jka} %author's name with academic degrees
\supervisor{Ing. Luk{\' a}{\v s} Brchl}
\acknowledgements{I would like to thank my family, my girlfriend and friends for support while writing this thesis and all my studies.
I also would like to thank my supervisor Ing. Luk{\' a}{\v s} Brchl, for this chance to realize me in mobile application development.}

%----------------------------------------------------------------------------------------------------------------------------------------

\abstractEN{This master's thesis focuses on cross-platform mobile application development in the Flutter framework for safer drone operations.
This thesis contains full application analysis, design, and implementation. Its main functionality is the management of drones and identification devices attached to drones, showing restricted flight zones, and planning and watching drone flights.

The analysis emphasizes the analysis of the existing web platform and all infrastructure, which provides data to the client and third-party applications.
The design contains a description of the application structure and used architecture patterns of the Flutter framework.
The implementation describes details about realization and contains detailed instructions to launch in the development, test, and production environment.
Concurrently, it contains a description of the necessary configuration files that differs across the various environment and have security disposition.}

%----------------------------------------------------------------------------------------------------------------------------------------

\abstractCS{Tato diplomov{\' a} pr{\' a}ce se zab{\' y}v{\' a} v{\' y}vojem multiplatformn{\' i} mobiln{\' i} aplikace ve frameworku Flutter pro bezpe{\v c}n{\v e}j{\v s}{\' i} provoz dron{\r u}.
Pr{\' a}ce obsahuje kompletn{\' i} anal{\' y}zu, n{\' a}vrh a samotnou implementaci aplikace, jej{\' i}{\v z} hlavn{\' i} fukcionalitou je spr{\' a}va dron{\r u} a identifika{\v c}n{\' i}ch za{\v r}{\' i}zen{\' i} p{\v r}ipevn{\v e}n{\' y}ch k dron{\r u}m, zobrazov{\' a}n{\' i} zak{\' a}zan{\' y}ch z{\' o}n pro lety a pl{\' a}nov{\' a}n{\' i} a sledov{\' a}n{\' i} let{\r u} dron{\r u}.

V anal{\' y}ze je kladen d{\r u}raz na rozbor existuj{\' i}c{\' i} webov{\' e} platformy a cel{\' e} infrastruktury, kter{\' a} poskytuje data klientsk{\' y}m aplikac{\' i}m a aplikac{\' i}m t{\v r}et{\' i}ch stran.
N{\' a}vrh obsahuje popis struktury aplikace a pou{\v z}it{\' e} architektonick{\' e} vzory frameworku Flutter.
Implementace popisuje detaily o realizaci a obsahuje podrobn{\' y} n{\' a}vod pro spu{\v s}t{\v e}n{\' i} ve v{\' y}vojov{\' e}m, testovac{\' i}m a produk{\v c}n{\' i}m prost{\v r}ed{\' i}.
Z{\' a}rove{\v n} obsahuje popis d{\r u}le{\v z}it{\' y}ch konfigura{\v c}n{\' i}ch soubor{\r u}, kter{\' e} se li{\v s}{\' i} nap{\v r}{\' i}{\v c} r{\r u}zn{\' y}mi prost{\v r}ed{\' i}mi a maj{\' i} bezpe{\v c}nostn{\' i} charakter.}

%----------------------------------------------------------------------------------------------------------------------------------------

\placeForDeclarationOfAuthenticity{Prague}
\keywordsCS{Multiplatformn{\' i}, mobiln{\' i}, aplikace, dron, {\v r}{\' i}zen{\' i}, Flutter.}
\keywordsEN{Cross-platform, mobile, application, drone, operation, Flutter.}
\declarationOfAuthenticityOption{1} %select as appropriate, according to the desired license (integer 1-6)
% \website{http://site.example/thesis} %optional thesis URL


\begin{document}

% \newacronym{CVUT}{{\v C}VUT}{{\v C}esk{\' e} vysok{\' e} u{\v c}en{\' i} technick{\' e} v Praze}
% \newacronym{FIT}{FIT}{Fakulta informa{\v c}n{\' i}ch technologi{\' i}}


\setsecnumdepth{part}
\chapter{Introduction}\label{ch:introduction}
This thesis is focused on ... zajištění bezpečnosti provozu dronů.

%Since 2017, RLP is solving a problem with managing of flight drone operations.
%It is the reason why this thesis was established.

V roce 2012 se začali v ČR rozmáhat drony.
V té době to byly spíše hračky, které si hobby piloti sestavovali sami doma.
Proto je provázeli obvykle technické problémy.
Po určité době v nich začali vidět potenciál velské firmy a začali na nich stavět svou vizi.
Například společnost Amazon představila svůj koncept autonomních dronů schopných doručovat zboží z logistických skladů přímo k zákazníkům domů.

%TODO: todo
At first, I would like to talk about the solutions in the world especially using of drones in the industry.
For example, Amazon Inc., company is working on a proof of concept which can use drones to delivery a package ordered by a customer.
This news was published in Forbes magazine by Jillian D'Onfro.\cite{amazonArticle}
The customer could be more satisfied with this shopping because the time of delivering can be cut on as the tiny part of the time as now.
Even Amazon has decided to produce their own drone products.
One of them is Prime Air delivery drone.

To make the delivery process easier, Amazon is developing autonomous drones application to minimize costs.
Amazon using/usage/application

autonomous drones application

Potentially, it could be possible to use drones to delivery the medical materials in case of big accident.
Especially, after a hurricane, fight conflict, or even during epidemic diseases like Coronavirus.

Zvýšením počtů majitelů dronů začali vznikat problémy.
Většina hobby pilotů při létech porušovala zákon a ohrožovala bezpečnost leteckého provozu.
Důvodem byla neznalost patřičných zákonů, které do té doby museli znát pouze piloty posádkových letounů.
Proto se postupem času začal tento problém řešit.
Přestože se na piloty bezpilotních letounů vztahují i ty posádkových letounů.
Na druhou stranu, pilotům dronů stačí znát značení zón and restricted areas a nemusí být zatížení kompletní instruktáží jako jsou piloti posádkových letounů.

Pro bezpečný provoz je nutné znát základní typy a značení zón.
Tyto zóny se dělí na několik typů.
Jsou to CTR, TMA, airports, R, D, A.
Dále je u těchto zón důležitý parametr lower and upper altitude level.
Tyto hladiny (levels) je dělí do několika skupin.
Jsou to


%TODO
Drones problematics and advisories (rules)

%\cite{} http://www.laacr.cz/SiteCollectionDocuments/rozdeleni-vp/p03-%20nakolenik%202019%20vnejsek%20rez%20D2.jpg
Popsat letové hladiny a zóny


!!! Použito v abstraktu !!!
Tato práce se zabývá celým vývojem multiplatformní mobilní aplikace ve frameworku Flutter pro plánování a sledování letů.
Tento vývoj zahrnuje kompletní analýzu, návrh a samotnou implementaci.

V analýze je kladen důraz na rozbor existující webové platformy a celé infrastuktury, která poskytuje data klientským aplikacím a vystavuje API endpoint aplikacím třetích stran.

Návrh se dělí na architekturu a design mobilní aplikace z pohledu softwaru.
Zde je popsána celá struktura aplikace a využité architektonické patterny frameworku Flutter.

Implementace popisuje detaily o realizaci a obsahuje podrobný návod pro spuštění ve vývojovém, testovacím a produkčním prostředí.
Zároveň obsahuje popis důležitých konfiguračních souborů, které se liší napříč různými prostředími a mají bezpečnostní charakter.

\section{Motivation}\label{sec:motivation}
Motivací podílet se na této práci pro mě byla v tom, že se mohu realizovat ve vývoji mobilních applikací.
Dalším důvodem byly úspěchy v soutěží dvou zakládajících členů společnosti Dronetag Luk{\' a}{\v s} Brchl and Marian Hlav{\' a}{\v c}.
Nejenže vyhráli hackathon, ale dokonce se umístili v soutěžích na evropské a světové úrovni.
Motivící pro mě bylo také možnost podílet se na produktu budoucnosti, který bude mít v velké využití a už nyní má velký potenciál.
Má svůj důvod, proč se trhu s drony říká exponenciální.
%TODO
využití dronů v průmyslu + trochu historie
Exponenciální trh

úspěchy Dronetagu v soutěžích, Hackathonech


\section{Objectives}\label{sec:objectives}
%TODO
Cílem této práce je vyvinou cross-platform mobile application pro zajištění bezpečnosti dronů.
Aplikace bude implementovaná podle best-practice principů ve frameworku Flutter.
Zároveň tato práce obsahuje popis základních konceptů frameworku Flutter v kapitole Analysis of Flutter.
Implementace zahrnuje i částečné otestování unit testy a integrační testy a deployment specification.
Aplikace bude splňovat všechny funkční požadavky.
Těmi jsou vykreslování zón v mapě, zobrazování létajících dronů opatřených Dronetag zařízením v reálném čase, plánování letů a management vlastních letounů, zařízení a historie letů.

%\section{Problem statement}\label{sec:problem-statement}
%Drones using in industry increasing every year.
%Nowadays,


\setsecnumdepth{all}
\chapter{Related projects}

In the world...

\section{Project in Dronetag s.r.o. company}
TODO
\chapter{Dronetag backend infrastructure}\label{ch:dronetag-backend-infrastructure}
This chapter describes all Dronetag backend infrastructure and contains a detailed description of the web Droneteg platform.
It consists the whole technical stack that ensures data providing.
My co-workers had to determine the convenient and reliable way how to gradually construct a useful architecture of the infrastructure.
It starts with an IoT module and simplified web, and it continues to implement WebSockets with alone Live Service.
Current parts of the stack is following:
\begin{itemize}
    \item Backend,
    \item Frontend,
    \item OLP Message Broker,
    \item Live Service and
    \item .
\end{itemize}
Every web application in stack is deployed and managed by Docker. %\cite{}
It is the easiest way to develop and publish new version software.
Thanks to docker compose tool we can have a separated parts of the infrastructure.
In case of bigger changes, we have to change only one and others are without changes.

Due to security reasons, the web API is divided to Private and Public API endpoints.

In additional determination, this infrastructure is divided to Staging and Production environment.
Staging is for development and testing purposes and it usually runs on the same version as Production.
The exception is only when a new feature is establishing.

\section{Backend application}\label{sec:backend-application}
Python Django Web API app %TODO
It contains the Private and Public API endpoints.

\section{Database Model}\label{sec:database-model}
There is a database model to store data in Dronetag Backend.
The database model consists of followings entities:
\begin{itemize}
    \item User,
    \item Aircraft,
    \item Device,
    \item Flight,
    \item Telemetry measurement,
    \item Organization,
    \item Fleet,
    \item Aircraft vendor,
    \item Aircraft model,
    \item Airspace zone,
    \item User preference.
\end{itemize}

\subsection{User}\label{subsec:user}
This entity represents a user who signs up and logs in to the application.
It consists of user credentials and identification information.
Attributes of this entity are the following:
\begin{itemize}
    \item \textbf{E-mail} -- represents the e-mail that the user will be log in to the system and receive notifications,
    \item \textbf{Full name} -- represents optional full name of the user,
    \item \textbf{Password hash} -- represents a hashed password,
    \item \textbf{Phone number} -- represents a phone number that the user will be able to contact in case of emergency,
    \item \textbf{Country} -- represents a country where the user is used to fly.
\end{itemize}

\subsection{Aircraft}\label{subsec:aircraft}
This entity represents an aircraft that is connected with a Dronetag device.
Attributes of this entity are the following:
\begin{itemize}
    \item \textbf{Name} -- represents the aircraft name for easier recognition in My aircraft list,
    \item \textbf{\acrshort{uas} Operator ID} -- represents a unique code identifying a pilot who registered this aircraft in the \acrshort{uas}~(\acrlong{uas})~\cite{uas},
    \item \textbf{Weight} -- represents the weight of the aircraft.
\end{itemize}

\subsection{Device}\label{subsec:device}
This entity represents a physical device that sends live information to the Dronetag platform.
Attributes of this entity are the following:
\begin{itemize}
    \item \textbf{Serial number} -- represents a serial number of the device,
    \item \textbf{Name} -- represents the device name for easier recognition in My aircraft list,
    \item \textbf{Type} -- represents the model type of the device,
    \item \textbf{Last battery} -- represents the last battery value in Volts,
    \item \textbf{Last \acrshort{rsrp}} -- represents the \acrshort{rsrp}~(\acrlong{rsrp}) value.
\end{itemize}

\subsection{Flight}\label{subsec:flight}
This entity represents it represents a flight that a user has created.
Attributes of this entity are the following:
\begin{itemize}
    \item \textbf{Date planned start} -- represents a start date of planned the flight,
    \item \textbf{Date planned finish} -- represents a finish data of planned the flight,
    \item \textbf{Date started} -- represents a real start date of the flight,
    \item \textbf{Date finished} -- represents a real finish date of the flight,
    \item \textbf{Status} -- represents a flight status - values can be planned, current, finished and canceled,
    \item \textbf{Distance} -- represents a distance of the flight,
    \item \textbf{Duration} -- represents a duration of the flight,
    \item \textbf{Region \acrshort{geojson}} -- represents a reservation region in \acrshort{geojson} format~\cite{geoJson},
    \item \textbf{Max flight altitude} -- represents a maximum flight altitude,
    \item \textbf{Takeoff latitude} -- represents a latitude of a taking off position,
    \item \textbf{Takeoff longitude} -- represents a longitude of a taking off position,
    \item \textbf{Takeoff Geo altitude} -- represents a geological altitude of a taking off position,
    \item \textbf{Takeoff pressure} -- represents a taking off pressure,
    \item \textbf{Public} -- represents a boolean flag if the flight is public (visible for everyone).
\end{itemize}

\subsection{Telemetry Measurement}\label{subsec:telemetry-measurement}
This entity represents a telemetry measurement that the system receives and thanks so that a flight trajectory can be drawn.
Attributes of this entity are the following:
\begin{itemize}
    \item \textbf{Time} -- represents a date with the time of measurement,
    \item \textbf{Latitude} -- represents a measured latitude of a flying drone,
    \item \textbf{Longitude} -- represents a measured longitude of a flying drone,
    \item \textbf{Altitude} -- represents a measured altitude of a flying drone,
    \item \textbf{Geo altitude} -- represents a measured geological altitude of a flying drone,
    \item \textbf{Velocity X} -- represents a measured velocity in X-axis,
    \item \textbf{Velocity Y} -- represents a measured velocity in Y-axis,
    \item \textbf{Velocity Z} -- represents a measured velocity in Z-axis.
\end{itemize}

\subsection{Organization}\label{subsec:organization}
This entity represents an Organization for the fleet management functionality.
Attributes of this entity are the following:
\begin{itemize}
    \item \textbf{Name} -- represents the name of the organization,
    \item \textbf{Description} -- represents a text description of the organization.
\end{itemize}

\subsection{Fleet}\label{subsec:fleet}
This entity represents an organization's fleet management properties.
Attributes of this entity are the following:
\begin{itemize}
    \item \textbf{Name} -- represents a name of the fleet,
    \item \textbf{Color} -- represents a shown color of the fleet,
    \item \textbf{Deleted} -- represents a boolean flag if the fleet was deleted.
\end{itemize}

\subsection{Aircraft Vendor}\label{subsec:aircraft-vendor}
This entity represents an aircraft vendor who manufactures drones.
It contains only the one attribute name that represents the vendor name.

\subsection{Aircraft Model}\label{subsec:aircraft-model}
This entity represents an aircraft model that belongs to a vendor.
Attributes of this entity are the following:
\begin{itemize}
    \item \textbf{Name} -- represents the model name,
    \item \textbf{Weight} -- represents the weight of the model,
    \item \textbf{Vendor ID} -- represents a relationship to a Vendor.
\end{itemize}

\subsection{Airspace Zone}\label{subsec:airspace-zone}
This entity represents an airspace zone.
Attributes of this entity are the following:
\begin{itemize}
    \item \textbf{Name} -- represents a name of the zone,
    \item \textbf{Country} -- represents a country where the zone belongs,
    \item \textbf{Region \acrshort{geojson}} -- represents a region in \acrshort{geojson} format~\cite{geoJson}.
\end{itemize}

\subsection{User Preference}\label{subsec:user-preference}
This entity represents a user preference that is needed to store and share among various client applications.
Attributes of this entity are the following:
\begin{itemize}
    \item \textbf{Property} -- represents an identification of the property,
    \item \textbf{Value} -- represents the preference value.
\end{itemize}

\subsection{Live Service Database Model}\label{subsec:live-service-database-model}
During the development, it had been found out the current Backend is not sufficient for Dronetag needs.
So it was decided to divide the backend model into the Backend and Live Service model.
The Live Service contains only live real-time temporary data, so a Redis database was deployed.

"Redis is an open source (BSD licensed), in-memory data structure store, used as a database, cache and message broker.
It supports data structures such as strings, hashes, lists, sets, sorted sets with range queries, bitmaps, hyperloglogs, geospatial indexes with radius queries and streams.
Redis has built-in replication, Lua scripting, LRU eviction, transactions and different levels of on-disk persistence, and provides high availability via Redis Sentinel and automatic partitioning with Redis Cluster."~\cite{redis}
It means that the Redis is real-time storage that persists data only for a short time.
That is the reason why it is suitable for this purpose.
The Live Service database model consists of a Device and Telemetry entity.



\section {Broker}\label{sec:broker}
TODO

\chapter{Analysis of existing web and mobile applications}

TODO

\section{AirMap}
AirMap is the~main concurent application comparable with Dronetag platform.%\cite{}

\section{AisView}
TODO

\section{Fly carefully (Létejte zodpovědně)}
TODO


TODO integration to foreign API endpoints


\chapter{Analysis of Flutter}\label{ch:analysis-of-flutter}

This chapter describes the key reasons why Dronetag chose Flutter for the mobile application development.

Nowadays, many companies have more teams to maintain their application that ensure the companies' business.
Thanks that the costs are higher than could be and this is the reason why cross-platform frameworks were created.
There are many cross-platform mobile application platforms to develop an application like this.
The name of the main used frameworks are:
\begin{itemize}
    \item React Native,
    \item Ionic and
    \item Xamarin.
\end{itemize}
%TODO: přidat srovnání všech 4 frameworků

Flutter is a framework and was established by Google Inc., in 2017 and since that time, its popularity is still increasing.

React Native looks like the main competitor and is based on JavaScript.\cite{flutterVsReactNativeNevercodeIo}
Ionic has not a such good performance like Flutter and React Native.
Xamarin is based on C\# from Microsoft and thus we avoided it.
Flutter is based on Dart which was introduced by Google Inc., in 2011.
It is type safe and object oriented programming language similar to Java.\cite{dartTypeSystem}
Between advantages of Flutter belong:
\begin{itemize}
    \item Hot Reload, i.e., allows fast coding,
    \item One codebase: Development for two mobile platforms,
    \item Up to 50\% less testing,
    \item Faster app development,
    \item User-friendly designs,
    \item Perfect for MVPs and
    \item Less Code.\cite{flutterVsReactNativeHackrIo}
\end{itemize}

\section{General concept}\label{sec:general-concept}
"Flutter is an app SDK for building high-performance, high-fidelity apps for iOS, Android, web (beta), and desktop (technical preview) from a single codebase."\cite{flutterTechnicalOverview}
It means you need only one development team to reach an application for all platform in a single codebase.

"The goal is to enable developers to deliver high-performance apps that feel natural on different platforms.
We embrace differences in scrolling behaviors, typography, icons, and more."\cite{flutterTechnicalOverview}

%TODO: add some words of mine
Thanks this, we can style our application user interface on demand.

\begin{figure}
    \centering
    \begin{minipage}{.45\textwidth}
        \centering
        \includegraphics[width=.7\linewidth]{assets/hero-shrine-ios.png}
        \caption{Flutter demo app on iOS \cite{flutterTechnicalOverview}}
        \label{fig:flutter-demo-app-ios}
    \end{minipage}%
    \hspace{.05\linewidth}
    \begin{minipage}{.45\textwidth}
        \centering
        \includegraphics[width=.7\linewidth]{assets/hero-shrine-android.png}
        \caption{Flutter demo app on Android \cite{flutterTechnicalOverview}}
        \label{fig:flutter-demo-app-android}
    \end{minipage}
    \label{fig:flutter-demo-app}
\end{figure}


\section{Flutter User Interface}\label{subsec:flutter-ui}
"Flutter includes a modern react-style framework, a 2D rendering engine, ready-made widgets, and development tools.
These components work together to help you design, build, test, and debug apps.
Everything is organized around a few core principles."\cite{flutterTechnicalOverview}
Flutter has a widget tree that renders widget into nested tree, and these widgets are covering themselves.
It does not matter if a widget from the widget tree is Container, List View, Image, Text, Animation or anything else.
Everything is represented by a widget or its descendents.
Addition important classes for User Interface are MaterialApp and Scaffold.
MaterialApp and Scaffold we will describe in Android specific UI widgets - Material Design \ref{sec:android-specific-ui-widgets} section.


\section{Widgets}\label{sec:widgets}
How the founders of Flutter say: "Everything is a Widget".
In the following part, it will be clarified this statement.

"Widgets are the basic building blocks of a Flutter app's user interface.
Each widget is an immutable declaration of part of the user interface.
Unlike other frameworks that separate views, view controllers, layouts, and other properties, Flutter has a consistent, unified object model: the widget.
A~widget can define:
\begin{itemize}
    \item a structural element (like a button or menu),
    \item a stylistic element (like a font or color scheme),
    \item an aspect of layout (like padding),
    \item and so on \textellipsis"~\cite{flutterTechnicalOverview}
\end{itemize}

So each component of UI in Flutter is a descendent of the widget.
Many classes that inherit from a widget, but there are the essential descendants.
They are the following:
\begin{itemize}
    \item StatelessWidget,
    \item StatefulWidget,
    \item InheritedWidget.
\end{itemize}


\subsection{StatelessWidget}\label{subsec:statelesswidget}
"A widget that does not require mutable state.

A stateless widget is a widget that describes part of the user interface by building a constellation of other widgets that describe the user interface more concretely.
The building process continues recursively until the description of the user interface is fully concrete (e.g., consists entirely of
\textit{RenderObjectWidgets}~\cite{renderObjectWidget}, which describe concrete \textit{RenderObjects}~\cite{renderObject})."~\cite{statelessWidget}
It means that StatelessWidget is a useful component to render graphics elements that do not change during their lives.
StatelessWidget is class with Constructor and build method.
When the widget tree is built, it calls the build method.

"Stateless widget are useful when the part of the user interface you are describing does not depend on anything other than the configuration information in the object itself and the BuildContext in which the widget is inflated.
For compositions that can change dynamically, e.g. due to having an internal clock-driven state, or depending on some system state, consider using StatefulWidget."~\cite{statelessWidget}


\subsection{StatefulWidget}\label{subsec:statefulwidget}
"A widget that has mutable state.

State is information that (1) can be read synchronously when the widget is built and (2) might change during the lifetime of the widget.
It is the responsibility of the widget implementer to ensure that the \textit{State}~\cite{state} is promptly notified when such state changes, using \textit{State.setState}~\cite{setState}."~\cite{statefulWidget}
It means that StatefulWidget is a useful component to render graphics elements that change until they are disposed of.
StatefulWidget initializes its state, which represents a space for storing data.
The State is class with Constructor, initState and build method.
The Constructor is called when the State is created.
The initState method is called before the widget tree is rendered, and the build method is called when the tree renders itself.

Due to this fact, the Stateful widget has worse performance than StatelessWidget, and often it is difficult to keep a sustainable design of a component.
On the other hand, it offers a convenient way to create a widget with few states that will not be changed in the future development cycle.

"A stateful widget is a widget that describes part of the user interface by building a constellation of other widgets that describe the user interface more concretely.
The building process continues recursively until the description of the user interface is fully concrete (e.g., consists entirely of
\textit{RenderObjectWidgets}~\cite{renderObjectWidget}, which describe concrete \textit{RenderObjects}~\cite{renderObject})."~\cite{statefulWidget}


"Stateful widgets are useful when the part of the user interface you are describing can change dynamically, e.g. due to having an internal clock-driven state, or depending on some system state.
For compositions that depend only on the configuration information in the object itself and the BuildContext in which the widget is inflated, consider using StatelessWidget."~\cite{statefulWidget}


\subsection{InheritedWidget}\label{subsec:inheritedwidget}
"Base class for widgets that efficiently propagate information down the tree.

To obtain the nearest instance of a particular type of inherited widget from a build context, use \textit{BuildContext.dependOnInheritedWidgetOfExactType}~\cite{dependOnInheritedWidgetOfExactType}.

Inherited widgets, when referenced in this way, will cause the consumer to rebuild when the inherited widget itself changes state."~\cite{inheritedWidget}
It means that InheritedWidget is a useful component to pass data and offer to reduce boilerplate if there are many widgets nested in themselves.
Thanks to the BuildContext class and of a method, you can easily get the value you add as an input variable.
So it means the widget is suitable for passing a huge amount of data.
It loses a need to copy many parameters in a large tree structure.


\section{Bloc}\label{sec:bloc}
Bloc is an abbreviation of Business Logic Component and allows separating an application into separate layers.\cite{bloc}

"The goal of this package is to make it easy to implement the BLoC Design Pattern (Business Logic Component).

This design pattern helps to separate presentation from business logic.
Following the BLoC pattern facilitates testability and reusability.
This package abstracts reactive aspects of the pattern allowing developers to focus on converting events into states."\cite{bloc}

Also, Bloc is library written by Felix Angelov, and the concept bases on Reactive Programming.\cite{bloc}
In a nutshell, a structure of the concept consists of the following layer:
\begin{itemize}
    \item model - data model of your domain,
    \item UI - user graphical interface and
    \item controller - it controls communication between data and UI layers.
\end{itemize}
Bloc represents the controller layer.
When a user clicks on a button, it throws an event action that Bloc detects it and generate a new state.

There are the followings parts in Bloc:
\begin{itemize}
    \item \textbf{Events} are the input to a Bloc.
    They are commonly UI events such as button presses.
    Events are added to the Bloc and then converted to States.
    \item \textbf{States} are the output of a Bloc.
    Presentation components can listen to the stream of states and redraw portions of themselves based on the given state (see BlocBuilder for more details).
    \item \textbf{Transitions} occur when an Event is added after mapEventToState has been called but before the Bloc's state has been updated.
    A Transition consists of the currentState, the event which was added, and the nextState.
    \item \textbf{BlocSupervisor} oversees Blocs and delegates to BlocDelegate.
    \item \textbf{BlocDelegate} handles events from all Blocs which are delegated by the BlocSupervisor.
    Can be used to intercept all Bloc events, transitions, and errors.
    \textbf{It is a great way to handle logging/analytics as well as error handling universally.}\cite{bloc}
\end{itemize}

\begin{figure}
    \centering
    \includegraphics[scale=0.4]{assets/bloc_architecture.png}
    \caption{Bloc architecture\cite{bloc}}
    \label{fig:bloc-architecture}
\end{figure}

\subsection{HydratedBloc}\label{subsec:hydratedbloc}
HydratedBloc works as well as common Bloc.
The difference is in data storage.
Hydrated Bloc allow us to store data through a JSON object.
So, every time when your application loads data, you must not wait and show user progress indicator, but you are able to show stored data immediately.
When you receive data, you will simply render it into a screen.

\section{iOS specific UI widgets - Cupertino library}\label{sec:ios-specific-ui-widgets}
%TODO: Cupertino is a library ...
Pro využití prvků určených pro platformu iOS se Flutter vývojáři rozhodli včlenit typické prvky pro tuto platformu ve formě knihovny.
The reason is, aby se dali snadno používat pro vývoj zaměřený na iOS devices.
Tato knihovna se nazývá \textbf{Cupertino} a název vznikl podle názvu sídla společnosti Apple Inc., in Silicon Valley.
%TODO: přeformulovat
Součástí knihovny jsou prvky, se kterými jsou uživatelé iOS devices zvyklý pracovat.
Tato knihovna obsahuje prvky:
\begin{itemize}
    \item Cupertino Action Sheet,
    \item Cupertino Activity Indicator,
    \item Cupertino Alert Dialog,
    \item Cupertino Button,
    \item Cupertino Context Menu,
    \item Cupertino Date Picker,
    \item Cupertino Dialog,
    \item Cupertino Navigation Bar,
    \item Cupertino Page Scaffold,
    \item Cupertino Picker,
    \item Cupertino Slider,
    \item Cupertino Switch and
    \item Cupertino Tab Scaffold.\cite{cupertino}
\end{itemize}


\section{Android specific UI widgets - Material Design}\label{sec:android-specific-ui-widgets}
\textbf{Material Design} is a concept that was introduced by Google company in 2014.\cite{materialDesignArticle}
Na tomto konceptu je postavené celé Flutter User Interface.
Material Design contains components that interactively builds blocks for creating a user interface.
The components are following:
\begin{itemize}
    \item App bar,
    \item Bottom navigation,
    \item Buttons,
    \item Floating Action button,
    \item Cards,
    \item Chips,
    \item Dialogs,
    \item Lists,
    \item Pickers,
    \item Progress indicators,
    \item Sliders and
    \item Tabs.\cite{materialDesign}
\end{itemize}

\textbf{Scaffold} represents a new rendered screen which allows to place various elements.
\textbf{Scaffold} is possibly fully customize and change by the requirements.
Pokud dbáme na jednoduchost, můžeme využít stávající rozhraní a definovat:
\begin{itemize}
    \item AppBar,
    \item floatingActionButton, and
    \item bottomNavigationBar.
\end{itemize}
\textbf{AppBar} represents header of the screen and usually contains a title and action buttons.
\textbf{FloatingActionButton} je koncept tlačítek, která uživatel nejvíce používá a proto jsou v palcové zoně.
\textbf{BottomNavigationBar} je koncept horizontálního menu ktere je také v palcové zoně in foooter of the screen and umožňuje vytvořit hlavní přehledné menu.

Tyto dvě knihovny jsou pro naší aplikaci zajímávé z toho důvodu, protože v User interface používáme tyto prvky podle platformy, na které je applikace spouštěna.
Důvod je jednoduchý.
Uživatel by měl dostat to, s čím je zvyklý pracovat.

\chapter{User interface design}\label{ch:user-interface-design}

Describe a difference between Lo-Fi and Hi-Fi prototype.

I did not a Lo-Fi prototype because Dronetag is Start-up, company with limit budget, so I focused only on the Hi-Fi.
When I was making Hi-Fi, I inspired by a few concurrency application, which shows permitted flight zones and danger areas.

Describe a difference between UI and UX.
Write about techniques and reasons to concentrate on this things.

\section{Hi-Fi prototype}\label{sec:hi-fi-prototype}
We made a Hi-Fi prototype in Adobe XD.
We focused to the simplicity and "briefness" and keep focus on the most important rules of design UI and UX.
These rules are:
\begin{itemize}
    \item
\end{itemize}

\subsection{Dashboard}\label{subsec:dashboard2}
This screen contains:
\begin{itemize}
    \item Map canvas,
    \item Search button,
    \item Profile button that represents Menu,
    \item My location button,
    \item Map layer button and
    \item Fly now button.
\end{itemize}
In addition, for a logged user it contains:
\begin{itemize}
    \item Device status (if the user has already added some),
\end{itemize}

\subsection{Intro screen}\label{subsec:intro-screen}

\subsection{Login screen}\label{subsec:login-screen}

\section{Usability testing}\label{sec:usability-testing}
Describe usability testing flow

Describe right zoom when a user chooses drone on Dashboard map...

\subsection{Results}\label{subsec:results}

\subsection{Main mistakes in UI design}\label{subsec:main-mistakes-in-ui-design}
\chapter{Software architecture}
Describe software architecture
TODO
\section{Dependencies}

\section{Bloc and Hydrated Bloc}
In the begining, we counted on all of BLoCs will be the classic one. After short time, we have found out the storable concept is more convenient for this purpose. So, some of following are HydratedBloc

\subsection{UserProfileBloc}

\subsection{AircraftOperationBloc}

\subsection{DeviceOperationBloc}

\section{JsonSerializable classes}


TODO
\chapter{Software design}
Describe software design

%-------------------------------------------------
\section{Common}
describe content
todo

%-------------------------------------------------
\section{IO}
todo

\subsection{Model}
Model is dedicated on objects that are produced by API endpoint of Dronetag backend.

\subsection{Respositories}
todo

\subsection{Services}
todo

%-------------------------------------------------
\section{UI}

\subsection{Aircrafts}
todo

\subsection{Common}
todo

\subsection{Dashboard}
todo

\subsection{Devices}
todo

\subsection{Flights}
todo

\subsection{Intro}
todo

\subsubsection{Login}
todo

\subsubsection{Registration}
todo

\subsection{Profile}
todo

%-------------------------------------------------
\section{Util}
todo

\subsection{Error}
todo

\subsection{Extensions}
todo

\subsection{Preferences}
todo

\chapter{Mobile application implementation}\label{ch:mobile-application-implementation}

Words about the implementation

\section{Build instructions}\label{sec:build-instructions}
TODO

\section{Configuration files}\label{sec:configuration-files}

\subsection{Develop environment}\label{subsec:develop-environment}
TODO

\subsection{Test environment (Staging)}\label{subsec:test-environment}
TODO

\subsection{Production environment}\label{subsec:production-environment}
TODO

\chapter{Code testing}
TODO

\section{Unit testing}
TODO

\section{Integration testing}
TODO

\section{Mocker}
Mocker for flight drone simulations.
\chapter{Evaluation}\label{ch:evaluation}
Concept evaluation and incorporation this app into Dronetag products.

\setsecnumdepth{part}
\chapter{Conclusion}\label{ch:conclusion}



\bibliographystyle{template/iso690}
\bibliography{mybibliographyfile}

\setsecnumdepth{all}
\appendix

\chapter{Acronyms}\label{ch:acronyms}
% \printglossaries
\begin{description}
	\item[API] Application Interface
	\item[BLoC] Business Logic Component
	\item[CD] Continuous Delivery
	\item[CI] Continuous Integration
	\item[DI] Dependency Injection
	\item[GUI] Graphical user interface
	\item[HTTP] Hypertext Transfer Protocol
	\item[IoC] Inversion of Control
	\item[JSON] Java Script Object Notation
	\item[POI] Point of Interest
	\item[SVG] Scalable Vector Graphics
	\item[UI] User Interface
	\item[URL] Uniform Resource Locator
	\item[UX] User Experience
	\item[XML] Extensible markup language
\end{description}


\chapter{SD card contents}\label{ch:sd-card-contents}

%change appropriately

\begin{figure}
	\dirtree{%
		.1 readme.txt\DTcomment{the file with SD card contents description}.
		.1 apk\DTcomment{the directory with executable APK file for Android}.
		.1 src\DTcomment{the directory of source codes}.
		.2 dronetag\DTcomment{implementation sources}.
		.2 thesis\DTcomment{the directory of \LaTeX{} source codes of the thesis}.
		.1 text\DTcomment{the thesis text directory}.
		.2 DP\_Mat{\v e}jka\_Jan\_2020.pdf\DTcomment{the thesis text in PDF format}.
	}\label{fig:sd-card-content-table}
\end{figure}

\end{document}


